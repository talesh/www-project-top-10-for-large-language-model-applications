% Options for packages loaded elsewhere
\PassOptionsToPackage{unicode}{hyperref}
\PassOptionsToPackage{hyphens}{url}
%
\documentclass[
]{article}
\usepackage{amsmath,amssymb}
\usepackage{iftex}
\ifPDFTeX
  \usepackage[T1]{fontenc}
  \usepackage[utf8]{inputenc}
  \usepackage{textcomp} % provide euro and other symbols
\else % if luatex or xetex
  \usepackage{unicode-math} % this also loads fontspec
  \defaultfontfeatures{Scale=MatchLowercase}
  \defaultfontfeatures[\rmfamily]{Ligatures=TeX,Scale=1}
\fi
\usepackage{lmodern}
\ifPDFTeX\else
  % xetex/luatex font selection
\fi
% Use upquote if available, for straight quotes in verbatim environments
\IfFileExists{upquote.sty}{\usepackage{upquote}}{}
\IfFileExists{microtype.sty}{% use microtype if available
  \usepackage[]{microtype}
  \UseMicrotypeSet[protrusion]{basicmath} % disable protrusion for tt fonts
}{}
\makeatletter
\@ifundefined{KOMAClassName}{% if non-KOMA class
  \IfFileExists{parskip.sty}{%
    \usepackage{parskip}
  }{% else
    \setlength{\parindent}{0pt}
    \setlength{\parskip}{6pt plus 2pt minus 1pt}}
}{% if KOMA class
  \KOMAoptions{parskip=half}}
\makeatother
\usepackage{xcolor}
\setlength{\emergencystretch}{3em} % prevent overfull lines
\providecommand{\tightlist}{%
  \setlength{\itemsep}{0pt}\setlength{\parskip}{0pt}}
\setcounter{secnumdepth}{-\maxdimen} % remove section numbering
\usepackage{bookmark}
\IfFileExists{xurl.sty}{\usepackage{xurl}}{} % add URL line breaks if available
\urlstyle{same}
\hypersetup{
  hidelinks,
  pdfcreator={LaTeX via pandoc}}

\author{}
\date{}

\begin{document}

\subsection{LLM02: Gestione Non Sicura
dell'Output}\label{llm02-gestione-non-sicura-delloutput}

\subsubsection{Descrizione}\label{descrizione}

La Gestione Non Sicura dell'Output si riferisce nello specifico a una
validazione, sanificazione e gestione insufficiente degli output
generati da grandi modelli di linguaggio prima che vengano passati a
valle ad altri componenti e sistemi. Poiché il contenuto generato da un
LLM può essere controllato dal prompt in input, questo comportamento è
comparabile a fornire agli utenti un accesso indiretto a funzionalità
aggiuntive.

La Gestione Non Sicura dell'Output si differenzia dalla dipendenza
eccessiva (LLM09) in quanto si occupa degli output generati da un LLM
prima che vengano passati a valle, mentre la dipendenza eccessiva si
concentra su questioni più ampie riguardanti l'eccessiva fiducia
nell'accuratezza e nell'appropriatezza degli output di un LLM.

Un attacco che sfrutta la Gestione Non Sicura dell'Output può portare a
XSS e CSRF nei browser web, nonché a SSRF, escalation dei privilegi o
esecuzione di codice remoto (RCE) nei sistemi backend.

Le condizioni seguenti possono aumentare l'impatto di questa
vulnerabilità: * L'applicazione concede al LLM privilegi oltre a quelli
previsti per gli utenti finali, consentendo l'escalation dei privilegi o
l'esecuzione di codice remoto. * L'applicazione è vulnerabile ad
attacchi di iniezione di prompt indiretta, che potrebbero consentire a
un attaccante di ottenere l'accesso privilegiato all'ambiente di un
utente vittima. * Plugin di terze parti non validano adeguatamente gli
input.

\subsubsection{Esempi comuni di
vulnerabilità}\label{esempi-comuni-di-vulnerabilituxe0}

\begin{enumerate}
\def\labelenumi{\arabic{enumi}.}
\tightlist
\item
  L'output di un LLM viene inserito direttamente in una shell di sistema
  o in una funzione simile come exec o eval, causando l'esecuzione di
  codice remoto.
\item
  JavaScript o Markdown generati dal LLM vengono restituiti all'utente.
  Il codice viene quindi interpretato dal browser, causando un XSS.
\end{enumerate}

\subsubsection{Strategie di prevenzione e
mitigazione}\label{strategie-di-prevenzione-e-mitigazione}

\begin{enumerate}
\def\labelenumi{\arabic{enumi}.}
\tightlist
\item
  Trattare il modello come qualsiasi altro utente, adottando un
  approccio di zero-trust (fiducia zero), e applicare una corretta
  validazione degli input che vengono passati dal modello alle funzioni
  backend.
\item
  Seguire le linee guida OWASP ASVS (Application Security Verification
  Standard) per garantire una validazione e sanificazione efficace degli
  input.
\item
  Codificare l'output del modello che viene inviato agli utenti per
  mitigare l'esecuzione di codice indesiderato tramite JavaScript o
  Markdown. OWASP ASVS fornisce una guida dettagliata sulla codifica
  dell'output.
\end{enumerate}

\subsubsection{Esempi di scenari di
attacco}\label{esempi-di-scenari-di-attacco}

\begin{enumerate}
\def\labelenumi{\arabic{enumi}.}
\tightlist
\item
  Un'applicazione usa un plugin LLM per generare le risposte di un
  chatbot. Il plugin offre anche una serie di funzioni amministrative
  accessibili a un altro LLM privilegiato. Il LLM passa direttamente la
  sua risposta, senza una corretta validazione dell'output, al plugin
  causando l'arresto del plugin per manutenzione.
\item
  Un utente usa uno strumento di sintesi di siti web basato su un LLM
  per generare un riassunto conciso di un articolo. Il sito web include
  un'iniezione di prompt che istruisce il LLM a catturare contenuti
  sensibili dal sito web o dalla conversazione dell'utente. Il LLM può
  quindi codificare i dati sensibili e inviarli a un server controllato
  dall'attaccante, senza alcuna validazione o filtraggio dell'output.
\item
  Un LLM permette agli utenti di creare query SQL per un database nel
  backend attraverso una chat. Un utente richiede una query per
  eliminare tutte le tabelle del database. Se la query creata dal LLM
  non viene filtrata in nessun modo, allora tutte le tabelle del
  database verranno eliminate.
\item
  Un'applicazione web usa un LLM per generare contenuto a partire da
  prompt di testo inseriti dall'utente, senza sanificare l'output. Un
  attaccante potrebbe inviare un prompt creato ad arte che causa l'invio
  di un payload JavaScript non sanificato, portando a un XSS quando
  questo viene interpretato dal browser della vittima. La mancata
  validazione dei prompt rende possibile questo attacco.
\end{enumerate}

\subsubsection{Riferimenti e link
(Inglese)}\label{riferimenti-e-link-inglese}

\begin{enumerate}
\def\labelenumi{\arabic{enumi}.}
\tightlist
\item
  \href{https://security.snyk.io/vuln/SNYK-PYTHON-LANGCHAIN-5411357}{Arbitrary
  Code Execution}: \textbf{Snyk Security Blog}
\item
  \href{https://embracethered.com/blog/posts/2023/chatgpt-cross-plugin-request-forgery-and-prompt-injection./}{ChatGPT
  Plugin Exploit Explained: From Prompt Injection to Accessing Private
  Data}: \textbf{Embrace The Red}
\item
  \href{https://systemweakness.com/new-prompt-injection-attack-on-chatgpt-web-version-ef717492c5c2?gi=8daec85e2116}{New
  prompt injection attack on ChatGPT web version. Markdown images can
  steal your chat data.}: \textbf{System Weakness}
\item
  \href{https://embracethered.com/blog/posts/2023/ai-injections-threats-context-matters/}{Don't
  blindly trust LLM responses. Threats to chatbots}: \textbf{Embrace The
  Red}
\item
  \href{https://aivillage.org/large\%20language\%20models/threat-modeling-llm/}{Threat
  Modeling LLM Applications}: \textbf{AI Village}
\item
  \href{https://owasp-aasvs4.readthedocs.io/en/latest/V5.html\#validation-sanitization-and-encoding}{OWASP
  ASVS - 5 Validation, Sanitization and Encoding}: \textbf{OWASP AASVS}
\end{enumerate}

\end{document}
