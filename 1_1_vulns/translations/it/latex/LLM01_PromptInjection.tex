% Options for packages loaded elsewhere
\PassOptionsToPackage{unicode}{hyperref}
\PassOptionsToPackage{hyphens}{url}
%
\documentclass[
]{article}
\usepackage{amsmath,amssymb}
\usepackage{iftex}
\ifPDFTeX
  \usepackage[T1]{fontenc}
  \usepackage[utf8]{inputenc}
  \usepackage{textcomp} % provide euro and other symbols
\else % if luatex or xetex
  \usepackage{unicode-math} % this also loads fontspec
  \defaultfontfeatures{Scale=MatchLowercase}
  \defaultfontfeatures[\rmfamily]{Ligatures=TeX,Scale=1}
\fi
\usepackage{lmodern}
\ifPDFTeX\else
  % xetex/luatex font selection
\fi
% Use upquote if available, for straight quotes in verbatim environments
\IfFileExists{upquote.sty}{\usepackage{upquote}}{}
\IfFileExists{microtype.sty}{% use microtype if available
  \usepackage[]{microtype}
  \UseMicrotypeSet[protrusion]{basicmath} % disable protrusion for tt fonts
}{}
\makeatletter
\@ifundefined{KOMAClassName}{% if non-KOMA class
  \IfFileExists{parskip.sty}{%
    \usepackage{parskip}
  }{% else
    \setlength{\parindent}{0pt}
    \setlength{\parskip}{6pt plus 2pt minus 1pt}}
}{% if KOMA class
  \KOMAoptions{parskip=half}}
\makeatother
\usepackage{xcolor}
\setlength{\emergencystretch}{3em} % prevent overfull lines
\providecommand{\tightlist}{%
  \setlength{\itemsep}{0pt}\setlength{\parskip}{0pt}}
\setcounter{secnumdepth}{-\maxdimen} % remove section numbering
\usepackage{bookmark}
\IfFileExists{xurl.sty}{\usepackage{xurl}}{} % add URL line breaks if available
\urlstyle{same}
\hypersetup{
  hidelinks,
  pdfcreator={LaTeX via pandoc}}

\author{}
\date{}

\begin{document}

\subsection{LLM01: Iniezione di Prompt}\label{llm01-iniezione-di-prompt}

\subsubsection{Descrizione}\label{descrizione}

La vulnerabilità di tipo Iniezione di Prompt (inglese: prompt injection)
si verifica quando un attaccante manipola un modello linguistico di
grandi dimensioni (LLM) attraverso input fatti ad hoc, facendo in modo
che il LLM risponda inconsapevolmente alle intenzioni dell'attaccante.
Questo può essere fatto direttamente attraverso il ``jailbreaking''
(effrazione) del prompt di sistema, oppure indirettamente, manipolando
gli input esterni, portando potenzialmente all'esfiltrazione di dati,
all'ingegneria sociale e altre problematiche.

\begin{itemize}
\item
  \textbf{L'Iniezione di Prompt diretta}, conosciuta anche come
  ``jailbreaking'', si verifica quando un utente malintenzionato
  sovrascrive o rivela il prompt di sistema sottostante. Ciò può
  consentire agli attaccanti di sfruttare i sistemi backend interagendo
  con funzioni insicure e basi di dati accessibili tramite il LLM.
\item
  \textbf{L'Iniezione di Prompt indiretta} si verifica quando un LLM
  accetta input da fonti esterne che possono essere controllate da un
  attaccante, come siti web o file. L'attaccante può incorporare
  un'Iniezione di Prompt nel contenuto esterno, dirottando il contesto
  della conversazione. Ciò causerebbe una minore stabilità dell'output
  del LLM, consentendo all'attaccante di manipolare l'utente o i sistemi
  aggiuntivi a cui il LLM può accedere. Inoltre, le iniezioni di prompt
  indirette non hanno bisogno di essere visibili o leggibili da un
  umano, purché il testo venga analizzato dal LLM.
\end{itemize}

I risultati di un attacco di Iniezione di Prompt di successo possono
variare notevolmente - dalla richiesta di informazioni sensibili
all'influenza su processi decisionali critici sotto mentite spoglie di
normale funzionamento.

In attacchi avanzati, il LLM può essere manipolato per impersonare un
personaggio malevolo o interagire con plugin nell'ambiente dell'utente.
Ciò può portare alla divulgazione di dati sensibili, all'uso non
autorizzato di plugin o all'ingegneria sociale. In tali casi, il LLM
compromesso aiuta l'attaccante, aggirando i meccanismi di protezione
standard e mantenendo l'utente all'oscuro dell'intrusione. In questi
casi, il LLM compromesso agisce in sostanza come un agente per
l'attaccante, perseguendo i suoi obiettivi senza innescare i normali
meccanismi di protezione o senza segnalare l'intrusione all'utente
finale.

\subsubsection{Esempi comuni di
vulnerabilità}\label{esempi-comuni-di-vulnerabilituxe0}

\begin{enumerate}
\def\labelenumi{\arabic{enumi}.}
\tightlist
\item
  Un utente malintenzionato crea un'Iniezione di Prompt diretta per il
  LLM, ordinandogli di ignorare i prompt di sistema del creatore
  dell'applicazione e invece eseguire un prompt che restituisce
  informazioni private, pericolose o in generale sgradite.
\item
  Un utente usa un LLM per riassumere una pagina web contenente
  un'Iniezione di Prompt indiretta. Ciò fa sì che il LLM richieda
  informazioni sensibili all'utente e le esfiltri tramite JavaScript o
  Markdown.
\item
  Un utente malintenzionato carica un curriculum contenente un'Iniezione
  di Prompt indiretta. Il documento contiene un'Iniezione di Prompt con
  istruzioni per far sì che il LLM informi gli utenti che questo
  documento è eccellente, ad esempio un candidato perfettamente
  compatibile per un ruolo lavorativo. Un utente interno analizza il
  documento tramite il LLM per riassumerne il contenuto. In conseguenza
  all'Iniezione di Prompt, l'output del LLM indica che il documento è
  eccellente.
\item
  Un utente abilita un plugin collegato a un sito di e-commerce.
  Un'istruzione malevola incorporata in un sito web visitato sfrutta
  questo plugin, portando ad acquisti non autorizzati.
\item
  Un'istruzione e del contenuto malevoli, incorporati in un sito web
  visitato, sfruttano altri plugin per truffare gli utenti.
\end{enumerate}

\subsubsection{Strategie di prevenzione e
mitigazione}\label{strategie-di-prevenzione-e-mitigazione}

Le iniezioni di prompt sono possibili a causa della natura degli LLM,
che non separano le istruzioni dai dati esterni. Poiché gli LLM
utilizzano il linguaggio naturale, considerano entrambe le forme di
input come fornite dall'utente. Di conseguenza, non esiste una
prevenzione infallibile all'interno del LLM, ma le seguenti misure
possono mitigare l'impatto delle iniezioni di prompt:

\begin{enumerate}
\def\labelenumi{\arabic{enumi}.}
\tightlist
\item
  Applicare il controllo dei privilegi sull'accesso del LLM ai sistemi
  backend. Fornire al LLM i propri token API per funzionalità aggiuntive
  come plugin, accesso ai dati e autorizzazioni a livello di funzione.
  Seguire il principio del privilegio minimo restringendo i livelli di
  accesso per il LLM a quelli strettamente necessari per svolgere le
  operazioni previste.
\item
  Aggiungere un controllo umano (human in the loop) per funzionalità
  estese. Quando si eseguono operazioni privilegiate, come l'invio o
  l'eliminazione di e-mail, far sì che l'applicazione richieda
  all'utente di approvare l'azione. Ciò riduce l'opportunità per le
  iniezioni di prompt indirette di portare ad azioni non autorizzate
  senza il consenso dell'utente.
\item
  Separare il contenuto esterno dai prompt dell'utente. Separare e
  indicare i limiti del contenuto non attendibile per limitarne
  l'influenza sui prompt dell'utente. Ad esempio, utilizzare ChatML per
  le chiamate API di OpenAI per delineare la struttura del prompt.
\item
  Stabilire i confini di fiducia (trust boundary) tra il LLM, le fonti
  esterne e le funzionalità aggiuntive (per esempio plugin o funzioni a
  valle). Tuttavia, un LLM compromesso può comunque agire da
  intermediario (man-in-the-middle) tra le API dell'applicazione e
  l'utente, nascondendo o manipolando le informazioni prima di
  presentarle a quest'ultimo. Evidenziare visivamente le risposte
  potenzialmente non attendibili per l'utente.
\item
  Monitorare manualmente e periodicamente l'input e l'output del LLM,
  per verificare che sia conforme alle aspettative. Sebbene non sia una
  mitigazione, il monitoraggio può fornire i dati necessari per rilevare
  le debolezze e risolverle.
\end{enumerate}

\subsubsection{Esempi di scenario di
attacco}\label{esempi-di-scenario-di-attacco}

\begin{enumerate}
\def\labelenumi{\arabic{enumi}.}
\tightlist
\item
  Un attaccante effettua un'Iniezione di Prompt diretta a un chatbot di
  supporto basato su LLM. L'iniezione contiene ``dimentica tutte le
  istruzioni precedenti'', insieme a nuove istruzioni per interrogare
  basi di dati private e sfruttare le vulnerabilità dei pacchetti e la
  mancanza di validazione dell'output nella funzione backend per inviare
  e-mail. Ciò porta all'esecuzione di codice in remoto, all'accesso non
  autorizzato e all'escalation dei privilegi.
\item
  Un attaccante incorpora un'Iniezione di Prompt indiretta in una pagina
  web, istruendo il LLM a ignorare le istruzioni precedenti dell'utente
  e utilizzare un plugin LLM per eliminare le e-mail dell'utente. Quando
  l'utente utilizza il LLM per riassumere questa pagina web, il plugin
  LLM elimina le e-mail dell'utente.
\item
  Un utente usa un LLM per riassumere una pagina web il cui contenuto
  istruisce il modello a ignorare le precedenti istruzioni dell'utente e
  invece inserire un'immagine che rimanda a un URL che contiene un
  riassunto della conversazione. Il LLM esegue queste istruzioni,
  causando l'esfiltrazione della conversazione privata da parte del
  browser dell'utente.
\item
  Un utente malintenzionato carica un curriculum con un'Iniezione di
  Prompt. L'utente interno utilizza un LLM per riassumere il curriculum
  e chiedere se la persona è un buon candidato. A causa dell'Iniezione
  di Prompt, la risposta del LLM è sì, indipendentemente dal contenuto
  effettivo del curriculum.
\item
  Un attaccante invia un messaggio a un modello proprietario che si basa
  su un prompt di sistema, chiedendo al modello di ignorare le sue
  istruzioni precedenti e invece ripetere il suo prompt di sistema. Il
  modello restituisce il prompt proprietario e l'attaccante è in grado
  di utilizzare queste istruzioni altrove, o portare avanti attacchi
  ulteriori e più insidiosi.
\end{enumerate}

\subsubsection{Riferimenti e link
(Inglese)}\label{riferimenti-e-link-inglese}

\begin{enumerate}
\def\labelenumi{\arabic{enumi}.}
\tightlist
\item
  \href{https://simonwillison.net/2022/Sep/12/prompt-injection/}{Prompt
  injection attacks against GPT-3}: \textbf{Simon Willison}
\item
  \href{https://embracethered.com/blog/posts/2023/chatgpt-plugin-vulns-chat-with-code/}{ChatGPT
  Plugin Vulnerabilities - Chat with Code}: \textbf{Embrace The Red}
\item
  \href{https://embracethered.com/blog/posts/2023/chatgpt-cross-plugin-request-forgery-and-prompt-injection./}{ChatGPT
  Cross Plugin Request Forgery and Prompt Injection}: \textbf{Embrace
  The Red}
\item
  \href{https://arxiv.org/pdf/2302.12173.pdf}{Not what you've signed up
  for: Compromising Real-World LLM-Integrated Applications with Indirect
  Prompt Injection}: \textbf{Arxiv preprint}
\item
  \href{https://www.researchsquare.com/article/rs-2873090/v1}{Defending
  ChatGPT against Jailbreak Attack via Self-Reminder}: \textbf{Research
  Square}
\item
  \href{https://arxiv.org/abs/2306.05499}{Prompt Injection attack
  against LLM-integrated Applications}: \textbf{Arxiv preprint}
\item
  \href{https://kai-greshake.de/posts/inject-my-pdf/}{Inject My PDF:
  Prompt Injection for your Resume}: \textbf{Kai Greshake}
\item
  \href{https://github.com/openai/openai-python/blob/main/chatml.md}{ChatML
  for OpenAI API Calls}: \textbf{OpenAI Github}
\item
  \href{http://aivillage.org/large\%20language\%20models/threat-modeling-llm/}{Threat
  Modeling LLM Applications}: \textbf{AI Village}
\item
  \href{https://embracethered.com/blog/posts/2023/ai-injections-direct-and-indirect-prompt-injection-basics/}{AI
  Injections: Direct and Indirect Prompt Injections and Their
  Implications}: \textbf{Embrace The Red}
\item
  \href{https://research.kudelskisecurity.com/2023/05/25/reducing-the-impact-of-prompt-injection-attacks-through-design/}{Reducing
  The Impact of Prompt Injection Attacks Through Design}:
  \textbf{Kudelski Security}
\item
  \href{https://llm-attacks.org/}{Universal and Transferable Attacks on
  Aligned Language Models}: \textbf{LLM-Attacks.org}
\item
  \href{https://kai-greshake.de/posts/llm-malware/}{Indirect prompt
  injection}: \textbf{Kai Greshake}
\item
  \href{https://www.preamble.com/prompt-injection-a-critical-vulnerability-in-the-gpt-3-transformer-and-how-we-can-begin-to-solve-it}{Declassifying
  the Responsible Disclosure of the Prompt Injection Attack
  Vulnerability of GPT-3}: \textbf{Preamble; earliest disclosure of
  Prompt Injection}
\end{enumerate}

\end{document}
