% Options for packages loaded elsewhere
\PassOptionsToPackage{unicode}{hyperref}
\PassOptionsToPackage{hyphens}{url}
%
\documentclass[
]{article}
\usepackage{amsmath,amssymb}
\usepackage{iftex}
\ifPDFTeX
  \usepackage[T1]{fontenc}
  \usepackage[utf8]{inputenc}
  \usepackage{textcomp} % provide euro and other symbols
\else % if luatex or xetex
  \usepackage{unicode-math} % this also loads fontspec
  \defaultfontfeatures{Scale=MatchLowercase}
  \defaultfontfeatures[\rmfamily]{Ligatures=TeX,Scale=1}
\fi
\usepackage{lmodern}
\ifPDFTeX\else
  % xetex/luatex font selection
\fi
% Use upquote if available, for straight quotes in verbatim environments
\IfFileExists{upquote.sty}{\usepackage{upquote}}{}
\IfFileExists{microtype.sty}{% use microtype if available
  \usepackage[]{microtype}
  \UseMicrotypeSet[protrusion]{basicmath} % disable protrusion for tt fonts
}{}
\makeatletter
\@ifundefined{KOMAClassName}{% if non-KOMA class
  \IfFileExists{parskip.sty}{%
    \usepackage{parskip}
  }{% else
    \setlength{\parindent}{0pt}
    \setlength{\parskip}{6pt plus 2pt minus 1pt}}
}{% if KOMA class
  \KOMAoptions{parskip=half}}
\makeatother
\usepackage{xcolor}
\setlength{\emergencystretch}{3em} % prevent overfull lines
\providecommand{\tightlist}{%
  \setlength{\itemsep}{0pt}\setlength{\parskip}{0pt}}
\setcounter{secnumdepth}{-\maxdimen} % remove section numbering
\usepackage{bookmark}
\IfFileExists{xurl.sty}{\usepackage{xurl}}{} % add URL line breaks if available
\urlstyle{same}
\hypersetup{
  hidelinks,
  pdfcreator={LaTeX via pandoc}}

\author{}
\date{}

\begin{document}

\subsection{Introduzione}\label{introduzione}

L'introduzione sul mercato di massa dei chatbot pre-addestrati a fine
2022 ha innescato un'ondata di frenetico interesse per i modelli di
linguaggio a grandi dimensioni (LLM). Le aziende, desiderose di
sfruttare il potenziale degli LLM, li stanno integrando rapidamente nei
loro sistemi e nelle offerte destinate ai clienti. Tuttavia, la velocità
con cui gli LLM vengono adottati ha superato il tempo necessario per
stabilire protocolli di sicurezza esaustivi, lasciando molte
applicazioni vulnerabili a seri problemi di sicurezza.

Era evidente la necessità di una risorsa unificata per affrontare questi
problemi di sicurezza degli LLM. Gli sviluppatori, non sempre avvezzi ai
rischi associati agli LLM, si trovavano di fronte a risorse frammentate.
La missione di OWASP sembrava quindi perfetta per guidare un'adozione
sicura di questa tecnologia.

\subsubsection{A chi si rivolge questo
documento?}\label{a-chi-si-rivolge-questo-documento}

Il nostro pubblico principale sono gli sviluppatori, i data scientist e
gli esperti di sicurezza incaricati di pianificare e costruire
applicazioni e plugin basati su tecnologie LLM. Il nostro obiettivo è
fornire una guida pratica e concisa per aiutare questi professionisti a
muoversi nel terreno complesso e in continua evoluzione della sicurezza
degli LLM.

\subsubsection{La creazione della lista}\label{la-creazione-della-lista}

La creazione dell'OWASP Top 10 per le applicazioni LLM ha richiesto un
impegno significativo, realizzata grazie all'esperienza collettiva di un
gruppo internazionale di quasi 500 esperti, con più di 125 contributori
attivi. I nostri collaboratori provengono da contesti diversi, che
includono aziende nel campo dell'intelligenza artificiale, aziende del
settore della sicurezza, fornitori indipendenti di software, piattaforme
cloud e hyperscale, e il mondo della ricerca accademica.

Nel corso di un mese, abbiamo discusso e proposto potenziali
vulnerabilità e i membri del gruppo hanno considerato fino a 43 minacce
distinte. Attraverso molteplici round di selezione, abbiamo ridotto
queste proposte fino ad arrivare a una lista concisa delle 10
vulnerabilità più critiche.

Ognuna di queste vulnerabilità, congiuntamente agli esempi, ai
suggerimenti relativi alla prevenzione, agli scenari di attacco e ai
riferimenti, è stata ulteriormente esaminata e rifinita da sottogruppi
specializzati e sottoposta a una revisione pubblica, per assicurare che
la lista finale fosse il più possibile completa e concretamente
applicabile.

\subsubsection{Relazione con le altre liste OWASP Top
10}\label{relazione-con-le-altre-liste-owasp-top-10}

Anche se la nostra lista condivide il DNA con i tipi di vulnerabilità
che si possono trovare nelle altre liste OWASP Top 10, non ci limitiamo
a reiterarle, ma analizziamo le implicazioni uniche che queste
vulnerabilità hanno quando appaiono in applicazioni basate sugli LLM.

Il nostro obiettivo è di colmare la distanza tra i principi generali di
sicurezza delle applicazioni e le sfide specifiche poste dagli LLM.
Questo include l'esplorazione di come le vulnerabilità tradizionali
possano porre rischi differenti o possano essere sfruttate in nuovi modi
con gli LLM, e come i rimedi tradizionali debbano essere adattati alle
applicazioni basate sugli LLM.

\subsubsection{Riguardo alla versione
1.1}\label{riguardo-alla-versione-1.1}

Anche se la nostra lista condivide il DNA con i tipi di vulnerabilità
che si possono trovare nelle altre liste OWASP Top 10, non ci limitiamo
a reiterarle, ma analizziamo le implicazioni uniche che queste
vulnerabilità hanno quando appaiono in applicazioni basate sugli LLM.

Il nostro obiettivo è di colmare la distanza tra i principi generali di
sicurezza delle applicazioni e le sfide specifiche poste dagli LLM.
Questo include l'esplorazione di come le vulnerabilità tradizionali
possano porre rischi differenti o possano essere sfruttate in nuovi modi
con gli LLM, e come i rimedi tradizionali debbano essere adattati alle
applicazioni basate sugli LLM.

\subsubsection{Il futuro}\label{il-futuro}

La versione 1.1 di questa lista non sarà l'ultima. Ci aspettiamo di
aggiornare questa lista periodicamente, per stare al passo con
l'evoluzione del settore. Lavoreremo con la comunità per far evolvere la
tecnologia e creare altro materiale di studio per una serie di casi
d'uso. Miriamo inoltre a collaborare con gli organismi di
standardizzazione e i governi a riguardo della sicurezza
dell'intelligenza artificiale. Ti invitiamo a unirti al nostro gruppo e
contribuire.

\paragraph{Steve Wilson}\label{steve-wilson}

Responsabile del progetto OWASP Top 10 per le applicazioni LLM
\href{https://www.linkedin.com/in/wilsonsd/}{https://www.linkedin.com/in/wilsonsd}\\
Twitter/X: @virtualsteve

\paragraph{Ads Dawson}\label{ads-dawson}

Responsabile della release 1.1 e responsabile voci di vulnerabilità per
il progetto OWASP Top 10 per le applicazioni LLM
\href{https://www.linkedin.com/in/adamdawson0/}{https://www.linkedin.com/in/adamdawson0}
GitHub: @GangGreenTemperTatum

\subsubsection{Riguardo alla traduzione}\label{riguardo-alla-traduzione}

\textbf{Traduttori}

\begin{itemize}
\tightlist
\item
  \textbf{Fabrizio Cilli}\\
  \url{https://www.linkedin.com/in/fabriziocilli/}\strut \\
\item
  \textbf{Matteo Dora}\\
  \url{https://www.linkedin.com/in/mattbit/}\strut \\
\item
  \textbf{Riccardo Sirigu}\\
  \url{https://www.linkedin.com/in/riccardosirigu/}
\item
  \textbf{Valerio Alessandroni}\\
  \url{https://www.linkedin.com/in/valerio-alessandroni/}
\end{itemize}

Nella realizzazione di questa traduzione, abbiamo scelto consapevolmente
di impiegare solo traduttori umani, riconoscendo la natura
eccezionalmente tecnica e critica dell'OWASP Top Ten per gli LLM. I
traduttori elencati sopra non solo possiedono una profonda comprensione
del contenuto originale, ma anche la fluidità per rendere questa
traduzione un successo.

Talesh Seeparsan Responsabile traduzioni, OWASP Top 10 per le
applicazioni LLM \url{https://www.linkedin.com/in/talesh/}

\subsection{OWASP Top 10 per le applicazioni
LLM}\label{owasp-top-10-per-le-applicazioni-llm}

\subsubsection{LLM01: Iniezione di
Prompt}\label{llm01-iniezione-di-prompt}

Input artificiosi possono manipolare un modello linguistico di grandi
dimensioni, causando azioni non volute. Le iniezioni dirette
sovrascrivono i prompt di sistema, mentre quelle indirette manipolano
gli input provenienti da fonti esterne.

\subsubsection{LLM02: Gestione Non Sicura
dell'Output}\label{llm02-gestione-non-sicura-delloutput}

Questa vulnerabilità si manifesta quando l'output del LLM è accettato
senza previa verifica, esponendo i sistemi backend. L'abuso può portare
a conseguenze gravi come XSS, CSRF, SSRF, escalation dei privilegi o
esecuzione di codice remoto.

\subsubsection{LLM03: Avvelenamento dei Dati di
Apprendimento}\label{llm03-avvelenamento-dei-dati-di-apprendimento}

Questo si verifica quando i dati di apprendimento del LLM vengono
alterati, introducendo vulnerabilità o bias che ne compromettono la
sicurezza, l'efficacia o il comportamento etico. Le fonti di dati
includono Common Crawl, WebText, OpenWebText e libri.

\subsubsection{LLM04: Denial of Service del
Modello}\label{llm04-denial-of-service-del-modello}

Degli attaccanti causano operazioni che richiedono risorse elevate sui
modelli linguistici di grandi dimensioni, portando a degrado del
servizio o a costi elevati. La vulnerabilità è amplificata dalla natura
intensiva delle risorse degli LLM e dall'imprevedibilità degli input
dell'utente.

\subsubsection{LLM05: Vulnerabilità della
Supply-Chain}\label{llm05-vulnerabilituxe0-della-supply-chain}

Il ciclo di vita dell'applicazione LLM può essere compromesso da
componenti o servizi vulnerabili, portando ad attacchi di sicurezza.
L'utilizzo di dataset, modelli pre-addestrati e plugin di terze parti
può aggiungere altre vulnerabilità.

\subsubsection{LLM06: Divulgazione di Informazioni
Sensibili}\label{llm06-divulgazione-di-informazioni-sensibili}

Gli LLM possono rivelare dati confidenziali nelle risposte, portando ad
accessi non autorizzati, violazioni della privacy e falle di sicurezza.
Per mitigare questo rischio, è cruciale implementare un processo di
sanitizzazione dei dati e politiche utente rigorose.

\subsubsection{LLM07: Progettazione Insicura dei
Plugin}\label{llm07-progettazione-insicura-dei-plugin}

I plugin LLM possono avere input non sicuri e controlli di accesso
insufficienti. Questa mancanza di controllo dell'applicazione li rende
più facili da sfruttare e può comportare conseguenze come l'esecuzione
remota di codice.

\subsubsection{LLM08: Eccessiva
Autonomia}\label{llm08-eccessiva-autonomia}

I sistemi basati sugli LLM possono intraprendere azioni che conducono a
conseguenze non volute. Il problema nasce da funzionalità, permessi o
autonomia eccessivi concessi a questi sistemi.

\subsubsection{LLM09: Eccessivo
Affidamento}\label{llm09-eccessivo-affidamento}

Senza supervisione, sistemi o persone che fanno eccessivo affidamento
sugli LLM possono incorrere in disinformazione, malfunzionamenti,
problemi legali e vulnerabilità di sicurezza dovute a contenuti errati o
inappropriati generati dagli LLM.

\subsubsection{LLM10: Furto del Modello}\label{llm10-furto-del-modello}

Questa vulnerabilità consiste nell'accesso non autorizzato, la copia o
l'esfiltrazione di modelli LLM proprietari. L'impatto include perdite
economiche, compromissione del vantaggio competitivo e potenziale
accesso a informazioni sensibili.

\end{document}
