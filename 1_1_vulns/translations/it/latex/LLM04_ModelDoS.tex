% Options for packages loaded elsewhere
\PassOptionsToPackage{unicode}{hyperref}
\PassOptionsToPackage{hyphens}{url}
%
\documentclass[
]{article}
\usepackage{amsmath,amssymb}
\usepackage{iftex}
\ifPDFTeX
  \usepackage[T1]{fontenc}
  \usepackage[utf8]{inputenc}
  \usepackage{textcomp} % provide euro and other symbols
\else % if luatex or xetex
  \usepackage{unicode-math} % this also loads fontspec
  \defaultfontfeatures{Scale=MatchLowercase}
  \defaultfontfeatures[\rmfamily]{Ligatures=TeX,Scale=1}
\fi
\usepackage{lmodern}
\ifPDFTeX\else
  % xetex/luatex font selection
\fi
% Use upquote if available, for straight quotes in verbatim environments
\IfFileExists{upquote.sty}{\usepackage{upquote}}{}
\IfFileExists{microtype.sty}{% use microtype if available
  \usepackage[]{microtype}
  \UseMicrotypeSet[protrusion]{basicmath} % disable protrusion for tt fonts
}{}
\makeatletter
\@ifundefined{KOMAClassName}{% if non-KOMA class
  \IfFileExists{parskip.sty}{%
    \usepackage{parskip}
  }{% else
    \setlength{\parindent}{0pt}
    \setlength{\parskip}{6pt plus 2pt minus 1pt}}
}{% if KOMA class
  \KOMAoptions{parskip=half}}
\makeatother
\usepackage{xcolor}
\setlength{\emergencystretch}{3em} % prevent overfull lines
\providecommand{\tightlist}{%
  \setlength{\itemsep}{0pt}\setlength{\parskip}{0pt}}
\setcounter{secnumdepth}{-\maxdimen} % remove section numbering
\usepackage{bookmark}
\IfFileExists{xurl.sty}{\usepackage{xurl}}{} % add URL line breaks if available
\urlstyle{same}
\hypersetup{
  hidelinks,
  pdfcreator={LaTeX via pandoc}}

\author{}
\date{}

\begin{document}

\subsection{LLM04: Denial of Service del
Modello}\label{llm04-denial-of-service-del-modello}

\subsubsection{Descrizione}\label{descrizione}

Un attaccante può interagire con un LLM (Large Language Model) in modo
tale da causare un consumo eccezionalmente alto di risorse, risultando
in un deterioramento della qualità del servizio sia per sé stesso che
per altri utenti, e potenzialmente causando un aumento dei costi per le
risorse computazionali. Un altro aspetto critico per la sicurezza è la
possibilità che un attaccante interferisca o manipoli la ``finestra di
contesto'' di un LLM. Questa problematica sta diventando sempre più
rilevante a causa dell'uso crescente degli LLM in diverse applicazioni,
del loro intenso utilizzo di risorse, dell'imprevedibilità degli input
degli utenti e di una generale mancanza di consapevolezza tra gli
sviluppatori riguardo a questa vulnerabilità. Negli LLM, la finestra di
contesto rappresenta la lunghezza massima di testo che il modello può
gestire, includendo sia l'input che l'output. Questa caratteristica è
fondamentale per gli LLM, poiché determina la complessità dei costrutti
linguistici che il modello può comprendere e la quantità di testo che
può elaborare in un dato momento. La dimensione della finestra di
contesto è definita dall'architettura del modello e può variare a
seconda del modello specifico utilizzato.

\subsubsection{Esempi comuni di
vulnerabilità}\label{esempi-comuni-di-vulnerabilituxe0}

\begin{enumerate}
\def\labelenumi{\arabic{enumi}.}
\tightlist
\item
  Porre domande (prompts) che inducono a un utilizzo ripetitivo delle
  risorse attraverso la creazione di un elevato numero di compiti in
  coda, ad esempio impiegando strumenti come LangChain o AutoGPT.
\item
  Inviare interrogazioni insolitamente gravose in termini di risorse,
  utilizzando ortografie o sequenze di testo non convenzionali.
\item
  Sovraccarico continuo dell'input: un attaccante invia costantemente al
  LLM un flusso di input che eccede la finestra di contesto del modello,
  causando un consumo eccessivo di risorse computazionali.
\item
  Input lunghi ripetuti: l'attante invia al LLM input estesi in modo
  ripetitivo, ciascuno dei quali supera la capacità della finestra di
  contesto.
\item
  Espansione ricorsiva del contesto: l'attaccante formula un input che
  provoca un'espansione ricorsiva del contesto, obbligando il LLM a
  ingrandire e processare più volte la finestra di contesto.
\item
  Inondazione di input di lunghezza variabile: l'attaccante inonda il
  LLM con una grande quantità di input di lunghezze diverse, ognuno
  progettato per sfiorare il limite massimo della finestra di contesto.
  Questa tattica mira a sfruttare le inefficienze nella gestione degli
  input di dimensioni variabili, mettendo a dura prova il LLM e
  potenzialmente causandone il blocco.
\end{enumerate}

\subsubsection{Strategie di prevenzione e
mitigazione}\label{strategie-di-prevenzione-e-mitigazione}

\begin{enumerate}
\def\labelenumi{\arabic{enumi}.}
\tightlist
\item
  Implementare un controllo dell'input per assicurare che gli input
  degli utenti rispettino limiti predefiniti e siano privi di contenuti
  malevoli.
\item
  Impostare limiti alle risorse utilizzabili per ogni richiesta o
  passaggio, rallentando così l'esecuzione delle richieste più
  complesse.
\item
  Applicare limiti di frequenza alle chiamate API per contenere il
  numero di richieste che un singolo utente o indirizzo IP può fare
  entro un determinato lasso di tempo.
\item
  Limitare il numero di azioni in coda e il numero totale di azioni in
  un sistema terzo che interagisce con il LLM.
\item
  Monitorare continuamente l'utilizzo delle risorse del LLM per
  identificare picchi o schemi anomali che potrebbero indicare un
  attacco DoS.
\item
  Impostare limiti rigorosi sugli input in base alla finestra di
  contesto del LLM, per prevenire eccessivi sovraccarichi e
  l'esaurimento delle risorse.
\item
  Aumentare la consapevolezza tra gli sviluppatori sulle potenziali
  vulnerabilità ai DoS negli LLM e fornire indicazioni per
  un'implementazione sicura degli stessi.
\end{enumerate}

\subsubsection{Esempi di scenari di
attacco}\label{esempi-di-scenari-di-attacco}

\begin{enumerate}
\def\labelenumi{\arabic{enumi}.}
\tightlist
\item
  Un attaccante invia ripetutamente molteplici richieste complesse e
  onerose a un modello in hosting, portando a un deterioramento del
  servizio per gli altri utenti e a un aumento dei costi relative ai
  consumi delle risorse, nel servizio di hosting.
\item
  Durante l'elaborazione di un testo su una pagina web da parte di uno
  strumento basato su LLM che risponde a una query benigna, lo strumento
  si imbatte in un testo specifico e inizia a richiedere un numero
  eccessivo di pagine web, comportando un alto consumo di risorse.
\item
  Un attaccante bombarda continuamente il LLM con input che superano la
  sua finestra di contesto. L'attaccante può utilizzare script
  automatizzati o strumenti per inviare un alto volume di input,
  sovraccaricando le capacità di elaborazione del LLM. Di conseguenza,
  viene causato un eccessivo consumo di risorse computazionali, con
  conseguente rallentamento o inoperatività del sistema ospitante.
\item
  Un attaccante invia una serie di input sequenziali al LLM, in cui
  ciascun input è progettato per essere appena sotto il limite della
  finestra di contesto. Inviando ripetutamente questi input,
  l'attaccante mira a esaurire la capacità disponibile della finestra.
  Mentre il LLM fatica a elaborare ciascun input all'interno della sua
  finestra di contesto, le risorse del sistema si riducono
  drasticamente, risultando in un degrado delle prestazioni o in un
  completo diniego del servizio (DoS).
\item
  Un attaccante sfrutta i meccanismi ricorsivi del LLM per causare
  ripetutamente l'espansione del contesto. Creando un input che stimola
  il comportamento ricorsivo del LLM, l'attaccante costringe il modello
  a espandere e processare ripetutamente la finestra di contesto,
  consumando una significativa quantità di risorse computazionali.
  Questo attacco mette sotto sforzo il sistema e può portare a una
  condizione di DoS, rendendo il LLM non reattivo o causandone il fermo
  totale.
\item
  Un attaccante inonda il LLM con un grande volume di input di lunghezza
  variabile, costruiti specificamente per avvicinarsi o raggiungere il
  limite della finestra di contesto. Sopraffacendo il LLM con input di
  lunghezze variabili, l'attaccante mira a sfruttare qualsiasi
  inefficienza nell'elaborazione di questo tipo di input. Questo
  sovraccarico pesa eccessivamente sulle risorse del LLM, provocando un
  degrado delle prestazioni e ostacolando la capacità del sistema di
  rispondere a richieste legittime.
\item
  Mentre gli attacchi di tipo Denial of Service (DoS) puntano
  generalmente a sovraccaricare le risorse di un sistema, possono anche
  mirare ad altri aspetti del suo funzionamento, come sfruttare le
  vulnerabilità nelle limitazioni dell'API. Per esempio, in un recente
  episodio di sicurezza che ha coinvolto Sourcegraph, un attore malevolo
  ha acquisito un token di accesso amministrativo e lo ha usato per
  modificare i limiti di frequenza delle chiamate API. Questa azione
  avrebbe potuto causare interruzioni del servizio, abilitando volumi
  insolitamente alti di richieste.
\end{enumerate}

\subsubsection{Riferimenti e link
(Inglese)}\label{riferimenti-e-link-inglese}

\begin{enumerate}
\def\labelenumi{\arabic{enumi}.}
\tightlist
\item
  \href{https://twitter.com/hwchase17/status/1608467493877579777}{LangChain
  max\_iterations}: \textbf{hwchase17 on Twitter}
\item
  \href{https://arxiv.org/abs/2006.03463}{Sponge Examples:
  Energy-Latency Attacks on Neural Networks}: \textbf{Arxiv White Paper}
\item
  \href{https://owasp.org/www-community/attacks/Denial_of_Service}{OWASP
  DOS Attack}: \textbf{OWASP}
\item
  \href{https://lukebechtel.com/blog/lfm-know-thy-context}{Learning From
  Machines: Know Thy Context}: \textbf{Luke Bechtel}
\item
  \href{https://about.sourcegraph.com/blog/security-update-august-2023}{Sourcegraph
  Security Incident on API Limits Manipulation and DoS Attack}:
  \textbf{Sourcegraph}
\end{enumerate}

\end{document}
