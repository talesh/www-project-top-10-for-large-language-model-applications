% Options for packages loaded elsewhere
\PassOptionsToPackage{unicode}{hyperref}
\PassOptionsToPackage{hyphens}{url}
%
\documentclass[
]{article}
\usepackage{amsmath,amssymb}
\usepackage{iftex}
\ifPDFTeX
  \usepackage[T1]{fontenc}
  \usepackage[utf8]{inputenc}
  \usepackage{textcomp} % provide euro and other symbols
\else % if luatex or xetex
  \usepackage{unicode-math} % this also loads fontspec
  \defaultfontfeatures{Scale=MatchLowercase}
  \defaultfontfeatures[\rmfamily]{Ligatures=TeX,Scale=1}
\fi
\usepackage{lmodern}
\ifPDFTeX\else
  % xetex/luatex font selection
\fi
% Use upquote if available, for straight quotes in verbatim environments
\IfFileExists{upquote.sty}{\usepackage{upquote}}{}
\IfFileExists{microtype.sty}{% use microtype if available
  \usepackage[]{microtype}
  \UseMicrotypeSet[protrusion]{basicmath} % disable protrusion for tt fonts
}{}
\makeatletter
\@ifundefined{KOMAClassName}{% if non-KOMA class
  \IfFileExists{parskip.sty}{%
    \usepackage{parskip}
  }{% else
    \setlength{\parindent}{0pt}
    \setlength{\parskip}{6pt plus 2pt minus 1pt}}
}{% if KOMA class
  \KOMAoptions{parskip=half}}
\makeatother
\usepackage{xcolor}
\setlength{\emergencystretch}{3em} % prevent overfull lines
\providecommand{\tightlist}{%
  \setlength{\itemsep}{0pt}\setlength{\parskip}{0pt}}
\setcounter{secnumdepth}{-\maxdimen} % remove section numbering
\usepackage{bookmark}
\IfFileExists{xurl.sty}{\usepackage{xurl}}{} % add URL line breaks if available
\urlstyle{same}
\hypersetup{
  hidelinks,
  pdfcreator={LaTeX via pandoc}}

\author{}
\date{}

\begin{document}

\subsection{LLM07: Progettazione Insicura dei
Plugin}\label{llm07-progettazione-insicura-dei-plugin}

\subsubsection{Descrizione}\label{descrizione}

I plugin LLM sono estensioni che, una volta attivate, vengono invocate
automaticamente dal modello durante le interazioni con l'utente. La
gestione di questi plugin è affidata alla piattaforma che integra il
modello, e l'applicazione che lo utilizza potrebbe non avere il
controllo diretto sulla loro esecuzione, in particolare quando il
modello è gestito da un fornitore esterno. Inoltre, i plugin tendono ad
accettare input di testo libero dal modello senza alcuna validazione o
controllo sui tipi che gestisca le limitazioni sulla dimensione del
contesto. Ciò consente a un potenziale attaccante di formulare una
richiesta malevola al plugin, che potrebbe portare a una serie di
comportamenti indesiderati, inclusa l'esecuzione di codice remoto.

Il danno causato da input malevoli dipende spesso dai controlli di
accesso insufficienti e dalla mancata gestione nel tracciare le
autorizzazioni tra i diversi plugin. Un controllo di accesso inadeguato
permette a un plugin di fidarsi ciecamente di altri plugin e di
presumere che gli input ricevuti siano stati forniti dall'utente finale.
Tale controllo di accesso inadeguato può consentire agli input malevoli
di avere conseguenze dannose che vanno dall'esfiltrazione di dati,
all'esecuzione di codice remoto, fino all'acquisizione di privilegi non
autorizzati.

Questa sezione si concentra sulla creazione di plugin specifici per LLM
piuttosto che sui plugin di terze parti, coperti dalle Vulnerabilità
della Catena di Approvvigionamento del LLM
(Supply-Chain-Vulnerabilities).

\subsubsection{Esempi comuni di
vulnerabilità}\label{esempi-comuni-di-vulnerabilituxe0}

\begin{enumerate}
\def\labelenumi{\arabic{enumi}.}
\tightlist
\item
  Un plugin accetta tutti i parametri in un unico campo di testo anziché
  in parametri di input distinti.
\item
  Un plugin accetta stringhe di configurazione invece di parametri, che
  possono sovrascrivere intere impostazioni di configurazione.
\item
  Un plugin accetta comandi SQL o istruzioni di programmazione invece di
  parametri ristretti.
\item
  L'autenticazione è eseguita senza un'autorizzazione esplicita per un
  particolare plugin.
\item
  Un plugin tratta tutti i contenuti LLM come se fossero creati
  interamente dall'utente eseguendo qualsiasi azione richiesta senza
  chiedere ulteriori autorizzazioni.
\end{enumerate}

\subsubsection{Strategie di prevenzione e
mitigazione}\label{strategie-di-prevenzione-e-mitigazione}

\begin{enumerate}
\def\labelenumi{\arabic{enumi}.}
\tightlist
\item
  I Plugin dovrebbero accettare input che siano limitati e
  parametrizzati ove possibile e includere controlli sul tipo e la
  struttura dell'input. Dove non sia consentito, è opportuno inserire un
  secondo livello di chiamate fortemente tipizzate (controllo rigoroso
  sui tipi di parametro), analizzando le richieste e applicando
  validazione e sanificazione. Per gli input liberi, eventualmente
  necessari per alcune funzionalità dell'applicazione, è cruciale
  un'attenta revisione per prevenire l'invocazione di metodi dannosi.
\item
  Gli sviluppatori di plugin dovrebbero applicare le linee guida OWASP
  per gli standard di verifica della sicurezza applicativa (ASVS -
  Application Security Verification Standard) per assicurare adeguate
  validazione e sanitizzazione dell'input.
\item
  I Plugin dovrebbero essere ispezionati e testati in modo approfondito
  per assicurare adeguata validazione. Utilizzare sistemi di Scansione
  Statica del Codice (SAST) e Test Dinamici e Interattivi (DAST, IAST)
  all'interno dei processi (pipeline) di sviluppo.
\item
  I Plugin dovrebbero essere progettati con l'intento di minimizzare
  l'impatto dello sfruttamento di qualunque parametro di input insicuro
  come da linee guida sui Controlli di Accesso nello standard ASVS
  OWASP. Ciò prevede un controllo accessi che assicuri i privilegi
  strettamente necessari, e l'esposizione delle funzionalità
  strettamente necessarie al corretto funzionamento.
\item
  I Plugin dovrebbero utilizzare meccanismi di autorizzazione
  standardizzati come OAuth2, per consentire l'applicazione efficace dei
  controlli di autorizzazione e accesso. Inoltre le chiavi API
  dovrebbero essere utilizzate per fornire un contesto in cui applicare
  specifiche decisioni autorizzative che riflettano il flusso del plugin
  come chiaramente distinto da quello interattivo dell'utente.
\item
  Richiedere un intervento umano per l'autorizzazione e la conferma di
  ogni azione intrapresa da plugin particolarmente critici.
\item
  I Plugin sono tipicamente delle REST API, per cui si raccomanda agli
  sviluppatori l'applicazione delle raccomandazioni di cui alla lista
  OWASP Top 10 API Security Risks - 2023 per ridurre la presenza di
  vulnerabilità comuni.
\end{enumerate}

\subsubsection{Esempi di scenari di
attacco}\label{esempi-di-scenari-di-attacco}

\begin{enumerate}
\def\labelenumi{\arabic{enumi}.}
\tightlist
\item
  Un plugin accetta un URL base e istruisce il LLM nel combinare l'URL
  con una query per ottenere previsioni meteorologiche che sono incluse
  nell'elaborazione della richiesta dell'utente. Un utente
  malintenzionato può creare una richiesta in modo tale che l'URL punti
  verso un dominio sotto il suo controllo, permettendogli di iniettare
  il proprio contenuto nel modello LLM tramite il proprio dominio.
\item
  Un plugin accetta un input in forma libera in un unico campo che non
  viene validato. Un attaccante fornisce payload creati ad hoc per
  ottenere informazioni utili a partire dai messaggi di errore. Poi
  sfrutta vulnerabilità conosciute nelle dipendenze di terze parti per
  eseguire del codice arbitrario ed effettuando un'esfiltrazione di dati
  o aumentando i propri di privilegi.
\item
  Un plugin utilizzato per recuperare delle inclusioni (embedding) da un
  base di dati vettoriale accetta parametri di configurazione come
  stringa di connessione senza effettuarne la validazione. Ciò permette
  a un attaccante di sfruttare questa problematica e accedere ad altre
  basi di dati vettoriali cambiando nomi o parametri host ed esfiltrare
  rappresentazioni vettoriali a cui non dovrebbe avere accesso.
\item
  Un plugin accetta direttamente clausole SQL ``WHERE'' come parte di
  filtri avanzati, che vengono poi aggiunti alla query SQL. Ciò permette
  all'attaccante di eseguire un attacco di iniezione SQL.
\item
  Un attaccante utilizza l'iniezione indiretta di prompt per attaccare
  un plugin di gestione del codice insicuro, privo di validazione
  dell'input e con controlli di accesso deboli, per trasferire la
  proprietà del repository (archivio) e bloccare l'utente dai propri.
\end{enumerate}

\subsubsection{Riferimenti e link
(Inglese)}\label{riferimenti-e-link-inglese}

\begin{enumerate}
\def\labelenumi{\arabic{enumi}.}
\tightlist
\item
  \href{https://platform.openai.com/docs/plugins/introduction}{OpenAI
  ChatGPT Plugins}: \textbf{ChatGPT Developer's Guide}
\item
  \href{https://platform.openai.com/docs/plugins/introduction/plugin-flow}{OpenAI
  ChatGPT Plugins - Plugin Flow}: \textbf{OpenAI Documentation}
\item
  \href{https://platform.openai.com/docs/plugins/authentication/service-level}{OpenAI
  ChatGPT Plugins - Authentication}: \textbf{OpenAI Documentation}
\item
  \href{https://github.com/openai/chatgpt-retrieval-plugin}{OpenAI
  Semantic Search Plugin Sample}: \textbf{OpenAI Github}
\item
  \href{https://embracethered.com/blog/posts/2023/chatgpt-plugin-vulns-chat-with-code/}{Plugin
  Vulnerabilities: Visit a Website and Have Your Source Code Stolen}:
  \textbf{Embrace The Red}
\item
  \href{https://embracethered.com/blog/posts/2023/chatgpt-cross-plugin-request-forgery-and-prompt-injection./}{ChatGPT
  Plugin Exploit Explained: From Prompt Injection to Accessing Private
  Data}: \textbf{Embrace The Red}
\item
  \href{https://owasp-aasvs4.readthedocs.io/en/latest/V5.html\#validation-sanitization-and-encoding}{OWASP
  ASVS - 5 Validation, Sanitization and Encoding}: \textbf{OWASP AASVS}
\item
  \href{https://owasp-aasvs4.readthedocs.io/en/latest/V4.1.html\#general-access-control-design}{OWASP
  ASVS 4.1 General Access Control Design}: \textbf{OWASP AASVS}
\item
  \href{https://owasp.org/API-Security/editions/2023/en/0x11-t10/}{OWASP
  Top 10 API Security Risks -- 2023}: \textbf{OWASP}
\end{enumerate}

\end{document}
