% Options for packages loaded elsewhere
\PassOptionsToPackage{unicode}{hyperref}
\PassOptionsToPackage{hyphens}{url}
%
\documentclass[
]{article}
\usepackage{amsmath,amssymb}
\usepackage{iftex}
\ifPDFTeX
  \usepackage[T1]{fontenc}
  \usepackage[utf8]{inputenc}
  \usepackage{textcomp} % provide euro and other symbols
\else % if luatex or xetex
  \usepackage{unicode-math} % this also loads fontspec
  \defaultfontfeatures{Scale=MatchLowercase}
  \defaultfontfeatures[\rmfamily]{Ligatures=TeX,Scale=1}
\fi
\usepackage{lmodern}
\ifPDFTeX\else
  % xetex/luatex font selection
\fi
% Use upquote if available, for straight quotes in verbatim environments
\IfFileExists{upquote.sty}{\usepackage{upquote}}{}
\IfFileExists{microtype.sty}{% use microtype if available
  \usepackage[]{microtype}
  \UseMicrotypeSet[protrusion]{basicmath} % disable protrusion for tt fonts
}{}
\makeatletter
\@ifundefined{KOMAClassName}{% if non-KOMA class
  \IfFileExists{parskip.sty}{%
    \usepackage{parskip}
  }{% else
    \setlength{\parindent}{0pt}
    \setlength{\parskip}{6pt plus 2pt minus 1pt}}
}{% if KOMA class
  \KOMAoptions{parskip=half}}
\makeatother
\usepackage{xcolor}
\setlength{\emergencystretch}{3em} % prevent overfull lines
\providecommand{\tightlist}{%
  \setlength{\itemsep}{0pt}\setlength{\parskip}{0pt}}
\setcounter{secnumdepth}{-\maxdimen} % remove section numbering
\usepackage{bookmark}
\IfFileExists{xurl.sty}{\usepackage{xurl}}{} % add URL line breaks if available
\urlstyle{same}
\hypersetup{
  hidelinks,
  pdfcreator={LaTeX via pandoc}}

\author{}
\date{}

\begin{document}

\subsection{LLM08: Eccessiva Autonomia}\label{llm08-eccessiva-autonomia}

\subsubsection{Descrizione}\label{descrizione}

Un sistema basato su LLM è spesso dotato di una certa autonomia dallo
sviluppatore - la capacità di interfacciarsi con altri sistemi e
prendere iniziative in risposta a un prompt. La decisione su quali
funzioni invocare può anche essere delegata a un `agente' per
determinarla dinamicamente in base al prompt di input o all'output del
LLM.

L'Eccessiva Autonomia è la vulnerabilità che permette di intraprendere
azioni dannose in risposta a output inaspettati o ambigui generati da un
LLM (indipendentemente dalla causa del malfunzionamento del LLM; sia
essa allucinazione/confabulazione, iniezione di prompt
diretta/indiretta, plugin malevolo, prompt mal progettati, o
semplicemente un modello con prestazioni scadenti). La causa principale
dell'Eccessiva Autonomia è tipicamente una o più delle seguenti: troppe
funzionalità, permessi eccessivi o autonomia troppo elevata. Questo si
distingue dalla Gestione Insicura dell'Output, che si riferisce invece
alla mancanza di controlli adeguati sugli output generati dal LLM.

L'Eccessiva Autonomia può avere ripercussioni significative in termini
di riservatezza, integrità e disponibilità, variando ampiamente a
seconda dei sistemi con cui l'applicazione basata su LLM interagisce.

\subsubsection{Esempi comuni di
vulnerabilità}\label{esempi-comuni-di-vulnerabilituxe0}

\begin{enumerate}
\def\labelenumi{\arabic{enumi}.}
\tightlist
\item
  Funzionalità Eccessiva: Un agente LLM ha accesso a plugin che
  includono funzioni non strettamente necessarie. Ad esempio, uno
  sviluppatore deve concedere a un agente LLM la capacità di leggere
  documenti da un repository, ma il plugin di terze parti che sceglie di
  utilizzare consente anche la modifica o l'eliminazione dei documenti.
\item
  Funzionalità Eccessiva: Un plugin potrebbe essere stato utilizzato
  come prova durante una fase di sviluppo e poi scartato in favore di
  un'alternativa migliore, ma il plugin originale rimane disponibile
  all'agente LLM.
\item
  Funzionalità Eccessiva: Un plugin LLM con funzionalità aperte non
  riesce a filtrare adeguatamente le istruzioni di input consentendo
  l'esecuzione di comandi che superano le funzionalità previste
  dall'applicazione. Ad esempio, un plugin per eseguire un specifico
  comando shell non impedisce adeguatamente l'esecuzione di altri
  comandi shell.
\item
  Permessi Eccessivi: Un plugin LLM ha permessi su altri sistemi che non
  sono essenziali per il funzionamento previsto dell'applicazione. Ad
  esempio, un plugin progettato per leggere dati si connette a una base
  di dati utilizzando un'utenza che non ha solo permessi SELECT, ma
  anche UPDATE, INSERT e DELETE.
\item
  Permessi Eccessivi: Un plugin LLM progettato per operare per conto di
  un utente accede a sistemi a valle con un'utenza che possiede
  privilegi elevati. Ad esempio, un plugin per leggere l'archivio
  documenti dell'utente corrente si connette al repository dei documenti
  con un account privilegiato che ha accesso ai file di tutti gli
  utenti.
\item
  Autonomia Eccessiva: Un'applicazione o plugin basato su LLM non
  verifica e approva in modo indipendente azioni ad alto impatto. Ad
  esempio, un plugin che consente di eliminare i documenti di un utente
  esegue cancellazioni senza esplicita conferma da parte dell'utente.
\end{enumerate}

\subsubsection{Strategie di prevenzione e
mitigazione}\label{strategie-di-prevenzione-e-mitigazione}

Le seguenti azioni possono prevenire l'Eccessiva Autonomia:

\begin{enumerate}
\def\labelenumi{\arabic{enumi}.}
\tightlist
\item
  Consentire agli agenti LLM l'utilizzo di plugin/strumenti che offrono
  le funzioni strettamente necessarie al loro funzionamento. Ad esempio,
  se un sistema basato su LLM non richiede la capacità di recuperare i
  contenuti di un URL, tale plugin non dovrebbe essere offerto
  all'agente LLM.
\item
  Limitare le funzioni implementate nei plugin/strumenti LLM al minimo
  necessario. Ad esempio, un plugin che accede alla casella di posta
  elettronica di un utente per riassumere le email dovrebbe limitarsi
  alla capacità di leggere le email, quindi il plugin non dovrebbe
  includere altre funzionalità come l'eliminazione o l'invio di
  messaggi.
\item
  Evitare, dove possibile, funzioni aperte (ad esempio, eseguire un
  comando shell, recuperare un URL, ecc.) e usare plugin/strumenti con
  funzionalità più granulari. Ad esempio, un'app basata su LLM potrebbe
  aver bisogno di scrivere alcuni output su un file. Se ciò venisse
  implementato utilizzando un plugin per eseguire una funzione shell, la
  superficie d'attacco sarebbe molto più ampia (potrebbe essere eseguito
  qualsiasi altro comando shell). Un'alternativa più sicura potrebbe
  essere la realizzazione di un plugin per la scrittura di file che
  supporta solo quella specifica funzionalità.
\item
  Limitare i permessi concessi ai plugin/strumenti LLM verso altri
  sistemi ai minimi necessari per limitare l'ambito di azioni
  indesiderate. Ad esempio, un agente LLM che utilizza una base di dati
  di prodotti per fornire suggerimenti d'acquisto necessita solo
  dell'accesso in lettura alla tabella `prodotti'; non dovrebbe avere
  accesso ad altre tabelle, né la capacità di inserire, aggiornare o
  eliminare record. Ciò dovrebbe essere implementato applicando i
  permessi appropriati all'utenza che il plugin LLM utilizza per
  connettersi alla base di dati.
\item
  Monitorare le autorizzazioni dell'utente e il perimetro di sicurezza
  per garantire che le azioni intraprese per conto di un utente vengano
  eseguite sui sistemi a valle nel contesto previsto per quell'utente
  specifico e con i privilegi minimi necessari. Ad esempio, un plugin
  LLM che legge il repository di codice di un utente dovrebbe richiedere
  all'utente di autenticarsi tramite OAuth e con l'ambito minimo
  richiesto per lo scopo.
\item
  Utilizzare il controllo umano nel processo decisionale
  (human-in-the-loop) per richiedere l'approvazione umana di tutte le
  azioni prima che queste vengano intraprese. Questo può essere
  implementato in un sistema ``terzo'' (al di fuori dell'ambito
  dell'applicazione LLM) o all'interno del plugin/strumento LLM stesso.
  Ad esempio, un'app basata su LLM che crea e pubblica contenuti sui
  social media per conto di un utente dovrebbe includere una routine che
  preveda l'esplicita approvazione di quest'ultimo all'interno del
  plugin/strumento/API che effettua l'operazione di pubblicazione.
\item
  Implementare meccanismi di autorizzazione nei sistemi esterni
  piuttosto che lasciar decidere al modello LLM se un'azione sia
  consentita o meno. Quando si implementano strumenti/plugin, applicare
  il principio di mediazione assoluta dei flussi per assicurare che
  tutte le richieste fatte ai sistemi a valle tramite i plugin/strumenti
  vengano validate rispetto alle politiche di sicurezza.
\end{enumerate}

Le seguenti opzioni non prevengono l'Eccessiva Autonomia, ma possono
limitare il livello di danno causato:

\begin{enumerate}
\def\labelenumi{\arabic{enumi}.}
\tightlist
\item
  Registrare e monitorare l'attività dei plugin/strumenti LLM e dei
  sistemi da essi contattati per identificare eventuali azioni
  indesiderate, e rispondere di conseguenza.
\item
  Implementare un limite di frequenza (rate-limiting) per ridurre il
  numero di azioni indesiderate che possono verificarsi in un dato
  periodo di tempo, aumentando la probabilità di identificare azioni
  indesiderate tramite il monitoraggio prima che possano verificarsi
  danni significativi.
\end{enumerate}

\subsubsection{Esempi di scenari di
attacco}\label{esempi-di-scenari-di-attacco}

Un'app assistente personale basata su LLM ottiene l'accesso alla casella
di posta elettronica di un utente tramite un plugin per riassumere il
contenuto delle email in arrivo. Per ottenere questa funzionalità, il
plugin di posta elettronica necessita la capacità di leggere i messaggi,
tuttavia il plugin scelto dallo sviluppatore del sistema include anche
funzioni per l'invio di email. Il LLM è vulnerabile a un attacco di
iniezione indiretta di prompt, in cui un'email malevola induce il LLM a
comandare il plugin di posta elettronica per utilizzare la funzione
`invia messaggio' e inviare spam dalla casella di posta dell'utente.
Questo scenario potrebbe essere evitato: (a) eliminando la funzionalità
eccessiva utilizzando un plugin che si limita esclusivamente alla
lettura della posta, (b) eliminando i permessi eccessivi autenticandosi
al servizio email dell'utente tramite una sessione OAuth con privilegi
di sola lettura, e/o (c) eliminando l'autonomia eccessiva chiedendo
all'utente di approvare manualmente ogni invio di email redatto dal
plugin LLM. In aggiunta, il danno causato potrebbe essere mitigato
implementando un limite alla frequenza di invio sull'interfaccia che
spedisce la posta elettronica.

\subsubsection{Riferimenti e link
(Inglese)}\label{riferimenti-e-link-inglese}

\begin{enumerate}
\def\labelenumi{\arabic{enumi}.}
\tightlist
\item
  \href{https://embracethered.com/blog/posts/2023/chatgpt-cross-plugin-request-forgery-and-prompt-injection./}{Embrace
  the Red: Confused Deputy Problem}: \textbf{Embrace The Red}
\item
  \href{https://github.com/NVIDIA/NeMo-Guardrails/blob/main/docs/security/guidelines.md}{NeMo-Guardrails:
  Interface guidelines}: \textbf{NVIDIA Github}
\item
  \href{https://python.langchain.com/docs/modules/agents/tools/how_to/human_approval}{LangChain:
  Human-approval for tools}: \textbf{Langchain Documentation}
\item
  \href{https://simonwillison.net/2023/Apr/25/dual-llm-pattern/}{Simon
  Willison: Dual LLM Pattern}: \textbf{Simon Willison}
\end{enumerate}

\end{document}
