% Options for packages loaded elsewhere
\PassOptionsToPackage{unicode}{hyperref}
\PassOptionsToPackage{hyphens}{url}
%
\documentclass[
]{article}
\usepackage{amsmath,amssymb}
\usepackage{iftex}
\ifPDFTeX
  \usepackage[T1]{fontenc}
  \usepackage[utf8]{inputenc}
  \usepackage{textcomp} % provide euro and other symbols
\else % if luatex or xetex
  \usepackage{unicode-math} % this also loads fontspec
  \defaultfontfeatures{Scale=MatchLowercase}
  \defaultfontfeatures[\rmfamily]{Ligatures=TeX,Scale=1}
\fi
\usepackage{lmodern}
\ifPDFTeX\else
  % xetex/luatex font selection
\fi
% Use upquote if available, for straight quotes in verbatim environments
\IfFileExists{upquote.sty}{\usepackage{upquote}}{}
\IfFileExists{microtype.sty}{% use microtype if available
  \usepackage[]{microtype}
  \UseMicrotypeSet[protrusion]{basicmath} % disable protrusion for tt fonts
}{}
\makeatletter
\@ifundefined{KOMAClassName}{% if non-KOMA class
  \IfFileExists{parskip.sty}{%
    \usepackage{parskip}
  }{% else
    \setlength{\parindent}{0pt}
    \setlength{\parskip}{6pt plus 2pt minus 1pt}}
}{% if KOMA class
  \KOMAoptions{parskip=half}}
\makeatother
\usepackage{xcolor}
\setlength{\emergencystretch}{3em} % prevent overfull lines
\providecommand{\tightlist}{%
  \setlength{\itemsep}{0pt}\setlength{\parskip}{0pt}}
\setcounter{secnumdepth}{-\maxdimen} % remove section numbering
\usepackage{bookmark}
\IfFileExists{xurl.sty}{\usepackage{xurl}}{} % add URL line breaks if available
\urlstyle{same}
\hypersetup{
  hidelinks,
  pdfcreator={LaTeX via pandoc}}

\author{}
\date{}

\begin{document}

\section{OWASP Top 10 for LLM Version 1.1 Published: October 16,
2023}\label{owasp-top-10-for-llm-version-1.1-published-october-16-2023}

大規模言語モデル(LLM)への熱狂的な関心には目を見張るものがあります。これは、2022年後半にマスマーケット向けにリリースされた事前学習済みチャットボットをきっかけに起きている出来事です。LLMの可能性を活用しようとする企業は、LLMを自社の業務や顧客向けサービスに急速に組み込んでいます。ただ、猛スピードでLLMが採用される一方で、包括的にセキュリティを確保する手法が間に合っておらず、多くのアプリケーションがリスクの高い状況で、脆弱なままになっています。

LLMにおけるセキュリティ上の懸念に対処するためのまとまったリソースがない状況です。また、開発者は、LLMに特化したリスクに精通していない上に、リソースも散在したままになっています。そのような中、この技術を安全な方法で活用できるようにすることは、OWASPのミッションに完璧に適合すると考えています。

\subsection{対象読者}\label{ux5bfeux8c61ux8aadux8005}

このドキュメントの主な対象読者は、LLM技術を活用したアプリケーションやプラグインの設計・構築に携わる開発者、データサイエンティスト、そしてセキュリティ専門家です。目的は、このような専門家がLLMセキュリティの複雑で進化する領域を進む道案内となるよう、実践的で実用的、それでいて簡潔なセキュリティガイダンスを提供することを目指しています。

\subsection{The Making of the List
作成にあたって}\label{the-making-of-the-list-ux4f5cux6210ux306bux3042ux305fux3063ux3066}

OWASP Top 10 for LLMsリストの作成は、約 500
名の専門家からなる国際的なチームと、125
名以上の積極的な貢献者の専門知識を結集して構築された、大規模な事業でした。貢献者は、AI企業、セキュリティ企業、ISV、クラウド・ハイパースケーラー、ハードウェア・プロバイダ、そしてアカデミアを含む多様なバックグラウンドを持っています。

1ヶ月の間に、私たちはブレーンストーミングを行い、潜在的な脆弱性を提案し、チームメンバーは43の脅威を書き上げました。何度かの投票を繰り返すことにより、最も重要な10の脆弱性のシンプルなリストに絞り込みました。各脆弱性はその後、専門のサブチームによってさらにレビューして仕上げました。

各脆弱性は、よくある例、対策のヒント、攻撃シナリオ、参考文献があります。これらを専門のサブチームによってさらに精査・改良し、公開レビュを経て、できる限り包括的で実用的な最終リストを確立することとなりました。

\subsection{OWASP Top
10との関係}\label{owasp-top-10ux3068ux306eux95a2ux4fc2}

このリストは、他の OWASP トップ 10 リストに見られる脆弱性のタイプと DNA
を共有していますが、単に同じような脆弱性を繰り返すものではありません。むしろ、LLM
を利用するアプリケーションでこれらの脆弱性に遭遇した場合に、これらの脆弱性が持つユニークな意味を掘り下げています。

私たちの目標は、一般的なアプリケーション・セキュリティの原則と、LLM
がもたらす特有の課題との間の溝を埋めることです。これには、従来の脆弱性が
LLM
においてどのように異なるリスクをもたらすか、あるいは新しい方法で悪用される可能性があるか、また、従来の改善戦略を、
LLM
を利用するアプリケーションの場合にどのように適応させる必要があるかを探ります。

\subsection{Version 1.1
と将来について}\label{version-1.1-ux3068ux5c06ux6765ux306bux3064ux3044ux3066}

このリストのバージョン1.1は、私たちの最終版ではありません。私たちは、業界の現状に追いつくために、今後も更新を続けていくつもりです。私たちは、広範なコミュニティと協力し、最先端の技術を推し進め、さまざまな用途のための教材をさらに作成していきます。また、AIセキュリティに関するトピックについて、標準化団体や政府との協力も模索しています。私たちは、皆様が私たちのグループに参加し、活動に貢献されることを歓迎します。

Signature

Steve Wilson Project Lead, OWASP Top 10 for LLM Applications Twitter/X:
@virtualsteve

Ads Signature

Ads Dawson v1.1 Release Lead \& Vulnerability Entries Lead, OWASP Top 10
for LLM Applications LinkedIn: /in/adamdawson0 GitHub:
@GangGreenTemperTatum

\section{OWASP Top 10 for LLM}\label{owasp-top-10-for-llm}

\begin{itemize}
\tightlist
\item
  \textbf{LLM01: Prompt Injection プロンプトインジェクション}
  巧妙な入力によって大規模な言語モデル(LLM)を操作し、LLMが意図しない動作を引き起こします。直接注入はシステムのプロンプトを上書きし、間接注入は外部ソースからの入力を操作するものです。
\item
  \textbf{LLM02: Insecure Output Handling
  安全が確認されていない出力ハンドリング} この脆弱性は、LLM
  の出力が精査されずに受け入れられ、バックエンドシステムを露出させるという場合に起きることです。悪用されると、XSS、CSRF、SSRF、特権の昇格、リモート・コードの実行といった深刻な結果につながる可能性があります。
\item
  \textbf{LLM03: Training Data Poisoning 訓練データの汚染} LLM
  の訓練データが改ざんされ、セキュリティ、有効性、倫理的行動を損なう脆弱性やバイアスなどが入った状態です。情報源としては、Common
  Crawl、WebText、OpenWebText、書籍などがあります。
\item
  \textbf{LLM04: Model Denial of Service モデルのDoS}
  攻撃者はLLM上でリソースを大量に消費する操作を引き起こすことで、サービスの低下や高コストをもたらします。LLMはリソースを大量に消費し、ユーザーの入力が予測できないため、脆弱性は拡大します。
\item
  \textbf{LLM05: Supply Chain Vulnerabilities サプライチェーンの脆弱性}
  LLMアプリケーションのライフサイクルは、脆弱なコンポーネントやサービスによって侵害される可能性があり、セキュリティ攻撃につながります。サードパーティのデータセット、事前に訓練されたモデル、およびプラグインを使用することで、脆弱性が増える可能性があります。
\item
  \textbf{LLM06: Sensitive Information Disclosure 機微情報の漏えい}
  LLMは、その応答の中で不注意に機密データを暴露する可能性があり、不正なデータアクセス、プライバシー侵害、セキュリティ侵害につながります。これを軽減するためには、データのサニタイズと厳格なユーザー・ポリシーを導入することが極めて重要です。
\item
  \textbf{LLM07: Insecure Plugin Design
  安全が確認されていないプラグイン設計}
  LLMプラグインにおいて、入力の安全性が確認されておらず、あるいはアクセスコントロールが不十分である場合、このようなアプリケーションにおけるコントロールの欠如は、悪用が容易であり、リモート・コード実行のような結果をもたらす可能性があります。
\item
  \textbf{LLM08: Excessive Agency 過剰な代理行為}
  LLMベースのシステムは、意図しない結果を招く動作をすることがあります。この問題は、LLMベースのシステムに与えられた過剰な機能、権限、または自律性に起因します。
\item
  \textbf{LLM09: Overreliance 過度の信頼}
  十分監督されていないLLMに過度に依存したシステムや人々は、LLMが生成したコンテンツが不正確または不適切なものである場合、誤った情報、誤ったコミュニケーション、法的問題、セキュリティの脆弱性に直面する可能性があります。
\item
  \textbf{LLM10: Model Theft モデルの盗難}
  これには、独自のLLMモデルへの不正アクセス、コピー、または流出が含まれます。その影響には、経済的損失、競争上の優位性の低下、機密情報へのアクセスの可能性などが含まれます。
\end{itemize}

\end{document}
