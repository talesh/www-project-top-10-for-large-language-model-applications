% Options for packages loaded elsewhere
\PassOptionsToPackage{unicode}{hyperref}
\PassOptionsToPackage{hyphens}{url}
%
\documentclass[
]{article}
\usepackage{amsmath,amssymb}
\usepackage{iftex}
\ifPDFTeX
  \usepackage[T1]{fontenc}
  \usepackage[utf8]{inputenc}
  \usepackage{textcomp} % provide euro and other symbols
\else % if luatex or xetex
  \usepackage{unicode-math} % this also loads fontspec
  \defaultfontfeatures{Scale=MatchLowercase}
  \defaultfontfeatures[\rmfamily]{Ligatures=TeX,Scale=1}
\fi
\usepackage{lmodern}
\ifPDFTeX\else
  % xetex/luatex font selection
\fi
% Use upquote if available, for straight quotes in verbatim environments
\IfFileExists{upquote.sty}{\usepackage{upquote}}{}
\IfFileExists{microtype.sty}{% use microtype if available
  \usepackage[]{microtype}
  \UseMicrotypeSet[protrusion]{basicmath} % disable protrusion for tt fonts
}{}
\makeatletter
\@ifundefined{KOMAClassName}{% if non-KOMA class
  \IfFileExists{parskip.sty}{%
    \usepackage{parskip}
  }{% else
    \setlength{\parindent}{0pt}
    \setlength{\parskip}{6pt plus 2pt minus 1pt}}
}{% if KOMA class
  \KOMAoptions{parskip=half}}
\makeatother
\usepackage{xcolor}
\setlength{\emergencystretch}{3em} % prevent overfull lines
\providecommand{\tightlist}{%
  \setlength{\itemsep}{0pt}\setlength{\parskip}{0pt}}
\setcounter{secnumdepth}{-\maxdimen} % remove section numbering
\usepackage{bookmark}
\IfFileExists{xurl.sty}{\usepackage{xurl}}{} % add URL line breaks if available
\urlstyle{same}
\hypersetup{
  hidelinks,
  pdfcreator={LaTeX via pandoc}}

\author{}
\date{}

\begin{document}

\subsection{LLM08: 过度代理}\label{llm08-ux8fc7ux5ea6ux4ee3ux7406}

\subsubsection{描述}\label{ux63cfux8ff0}

基于LLM的系统通常由开发人员授予一定程度的代理权限 -
即与其他系统进行交互并在响应提示时执行操作的能力。关于调用哪些功能的决策也可以委托给LLM的``代理''根据输入提示或LLM的输出动态确定。

过度代理是一种漏洞,允许在LLM出现意外/模糊输出时执行破坏性操作(无论是什么导致LLM发生故障;无论是幻觉/混淆、直接/间接提示注入、恶意插件、工程不良的良性提示还是性能不佳的模型)。过度代理的根本原因通常是:功能过多、权限过多或自主权过多。这与不安全的输出处理不同,后者关注LLM输出的不充分审查。

过度代理可以导致涉及机密性、完整性和可用性等方面的一系列影响,这取决于LLM应用程序能够与哪些系统进行交互。

\subsubsection{常见的漏洞示例}\label{ux5e38ux89c1ux7684ux6f0fux6d1eux793aux4f8b}

\begin{enumerate}
\def\labelenumi{\arabic{enumi}.}
\tightlist
\item
  过度功能:LLM代理可以访问包括系统操作中不需要的功能在内的插件。例如,开发人员需要授予LLM代理从存储库中读取文档的能力,但他们选择使用的第三方插件还包括修改和删除文档的功能。
\item
  过度功能:在开发阶段可能会尝试使用某个插件,但最终选择了更好的替代方案,但原始插件仍然可供LLM代理使用。
\item
  过度功能:一个具有开放式功能的LLM插件未能正确过滤命令之外的输入指令,而这些命令对于应用程序的预期操作来说是不必要的。例如,用于运行一个特定shell命令的插件未能有效阻止其他shell命令的执行。
\item
  过度权限:LLM插件在其他系统上具有不必要的权限,这些权限对于应用程序的预期操作来说是不必要的。例如,一个旨在读取数据的插件连接到数据库服务器时,使用的身份不仅具有SELECT权限,还具有UPDATE、INSERT和DELETE权限。
\item
  过度权限:旨在代表用户执行操作的LLM插件使用通用高特权标识访问下游系统。例如,一个用于读取当前用户文档存储的插件连接到文档存储库时,使用了一个具有访问所有用户文件权限的特权帐户。
\item
  过度自主权:LLM基于应用程序或插件未能独立验证和批准高影响操作。例如,允许删除用户文档的插件执行删除操作时,无需用户的任何确认。
\end{enumerate}

\subsubsection{预防和缓解策略}\label{ux9884ux9632ux548cux7f13ux89e3ux7b56ux7565}

以下操作可以防止过度代理:

\begin{enumerate}
\def\labelenumi{\arabic{enumi}.}
\tightlist
\item
  限制LLM代理被允许调用的插件/工具,仅限于所需的最小功能。例如,如果LLM基础系统不需要获取URL内容的能力,那么不应该向LLM代理提供这样的插件。
\item
  限制在LLM插件/工具中实现的功能到最低程度。例如,一个插件用于访问用户的邮箱以总结电子邮件内容,可能只需要读取电子邮件的能力,因此插件不应包含其他功能,比如删除或发送邮件。
\item
  在可能的情况下避免开放式功能(例如运行shell命令、获取URL等),并使用更细粒度功能的插件/工具。例如,LLM基础应用程序可能需要将某些输出写入文件。如果使用插件运行shell功能来实现这一点,那么不希望的操作的范围就会非常大(可以执行任何其他shell命令)。更安全的替代方案是构建一个只支持特定功能的文件写入插件。
\item
  限制LLM插件/工具被授予的连接到其他系统的权限到最低程度,以限制不必要操作的范围。例如,使用产品数据库来向客户提供购买建议的LLM代理可能只需要对``产品''表的只读访问权限;它不应该具有对其他表的访问权限,也不应该具有插入、更新或删除记录的权限。这应该通过为LLM插件用于连接到数据库的身份应用适当的数据库权限来强制执行。
\item
  跟踪用户授权和安全范围,以确保代表用户执行的操作在特定用户的下游系统中以该特定用户的上下文和最低权限执行。例如,一个用于读取用户代码库的LLM插件应该要求用户通过OAuth进行身份验证,并且仅具有所需的最低权限。
\item
  利用人在循环控制,要求人在执行所有操作之前批准所有操作。这可以在下游系统中实现(超出了LLM应用程序的范围)或在LLM插件/工具本身中实现。例如,一个基于LLM的应用程序,用于代表用户创建和发布社交媒体内容,应该在插件/工具/API内包括一个用户批准程序,该程序执行``发布''操作。
\item
  在下游系统中实施授权,而不是依赖LLM来决定是否允许操作。在实现工具/插件时强制执行完整中介原则,以便通过工具/插件发出的所有请求都会根据安全策略进行验证。
\end{enumerate}

以下选项不能防止过度代理,但可以限制造成的损害程度:

\begin{enumerate}
\def\labelenumi{\arabic{enumi}.}
\tightlist
\item
  记录和监控LLM插件/工具和下游系统的活动,以识别不必要操作发生的地方,并采取相应的措施。
\item
  实施速率限制,减少在给定时间段内可以执行的不必要操作数量,增加通过监控发现不必要操作的机会,在造成重大损害之前发现问题。
\end{enumerate}

\subsubsection{攻击场景示例}\label{ux653bux51fbux573aux666fux793aux4f8b}

一个基于LLM的个人助手应用程序通过插件被授予访问个人邮箱的权限,以便总结收件箱中的邮件内容。为实现此功能,邮件插件需要具备读取邮件的能力,但系统开发人员选择使用的插件还包含发送邮件的功能。LLM容易受到间接提示注入攻击的威胁,恶意构造的入站邮件欺骗LLM,使其命令邮件插件调用``发送邮件''功能,从用户的邮箱发送垃圾邮件。可以通过以下方式避免这种情况:
(a) 通过使用仅提供邮件阅读功能的插件来消除过多的功能, (b)
通过使用OAuth会话进行身份验证,具有只读权限范围,来消除过多的权限,和/或
(c) 通过要求用户手动审核并点击LLM插件起草的每封邮件来消除过多的自主权。
或者,还可以通过在发送邮件接口上实施速率限制来减少造成的损害。

\subsubsection{参考链接}\label{ux53c2ux8003ux94feux63a5}

\begin{enumerate}
\def\labelenumi{\arabic{enumi}.}
\tightlist
\item
  \href{https://embracethered.com/blog/posts/2023/chatgpt-cross-plugin-request-forgery-and-prompt-injection./}{Embrace
  the Red: Confused Deputy Problem}: \textbf{Embrace The Red}
\item
  \href{https://github.com/NVIDIA/NeMo-Guardrails/blob/main/docs/security/guidelines.md}{NeMo-Guardrails:
  接口指南}: \textbf{NVIDIA Github}
\item
  \href{https://python.langchain.com/docs/modules/agents/tools/how_to/human_approval}{LangChain:
  工具的人工批准}: \textbf{Langchain文档}
\item
  \href{https://simonwillison.net/2023/Apr/25/dual-llm-pattern/}{Simon
  Willison: 双LLM模式}: \textbf{Simon Willison}
\end{enumerate}

\end{document}
