% Options for packages loaded elsewhere
\PassOptionsToPackage{unicode}{hyperref}
\PassOptionsToPackage{hyphens}{url}
%
\documentclass[
]{article}
\usepackage{amsmath,amssymb}
\usepackage{iftex}
\ifPDFTeX
  \usepackage[T1]{fontenc}
  \usepackage[utf8]{inputenc}
  \usepackage{textcomp} % provide euro and other symbols
\else % if luatex or xetex
  \usepackage{unicode-math} % this also loads fontspec
  \defaultfontfeatures{Scale=MatchLowercase}
  \defaultfontfeatures[\rmfamily]{Ligatures=TeX,Scale=1}
\fi
\usepackage{lmodern}
\ifPDFTeX\else
  % xetex/luatex font selection
\fi
% Use upquote if available, for straight quotes in verbatim environments
\IfFileExists{upquote.sty}{\usepackage{upquote}}{}
\IfFileExists{microtype.sty}{% use microtype if available
  \usepackage[]{microtype}
  \UseMicrotypeSet[protrusion]{basicmath} % disable protrusion for tt fonts
}{}
\makeatletter
\@ifundefined{KOMAClassName}{% if non-KOMA class
  \IfFileExists{parskip.sty}{%
    \usepackage{parskip}
  }{% else
    \setlength{\parindent}{0pt}
    \setlength{\parskip}{6pt plus 2pt minus 1pt}}
}{% if KOMA class
  \KOMAoptions{parskip=half}}
\makeatother
\usepackage{xcolor}
\setlength{\emergencystretch}{3em} % prevent overfull lines
\providecommand{\tightlist}{%
  \setlength{\itemsep}{0pt}\setlength{\parskip}{0pt}}
\setcounter{secnumdepth}{-\maxdimen} % remove section numbering
\usepackage{bookmark}
\IfFileExists{xurl.sty}{\usepackage{xurl}}{} % add URL line breaks if available
\urlstyle{same}
\hypersetup{
  hidelinks,
  pdfcreator={LaTeX via pandoc}}

\author{}
\date{}

\begin{document}

\subsection{LLM03: प्रशिक्षण डेटा
पॉइज़निंग}\label{llm03-ux92aux930ux936ux915ux937ux923-ux921ux91f-ux92aux907ux91cux928ux917}

\subsubsection{विवरण}\label{ux935ux935ux930ux923}

किसी भी मशीन लर्निंग मॉडल का प्रारंभिक बिंदु प्रशिक्षण डेटा है, बस ``कच्चा टेक्स्ट''।
अत्यधिक सक्षम होने के लिए (उदाहरण के लिए, भाषाई और विश्व ज्ञान रखने के लिए), इस
टेक्स्ट में क्षेत्र, शैलियों और भाषाओं की एक विस्तृत श्रृंखला होनी चाहिए। एक बड़ा भाषा
मॉडल प्रशिक्षण डेटा से सीखे गए पैटर्न के आधार पर आउटपुट उत्पन्न करने के लिए गहरे न्यूरल
नेटवर्क का उपयोग करता है।

प्रशिक्षण डेटा विषाक्तता से तात्पर्य है की ,डेटा को इस प्रकार से व्यवस्थित करना की
कमजोरियों, पिछले दरवाजे या कोई सुरक्षा बायस को ढूंढ सके । यह विषाक्तताये मॉडल की
सुरक्षा, प्रभावशीलता तथा नैतिक व्यवहार को जोखिम मे डाल सकती है। विषाक्त
जानकारीया यूज़र को दी जा सकती है जिससे कार्य के प्रदर्शन गिरावट, डाउनस्ट्रीम
सॉफ़्टवेयर का शोषण और प्रतिष्ठा को नुकसान जैसे अन्य जोखिम पैदा हो सकते हैं। भले ही यूज़र
समस्यागृत AI आउटपुट पर अविश्वास करते हों, लेकिन जोखिम बने रहते हैं, जिनमें मॉडल की
कमज़ोर क्षमताएं और ब्रांड की प्रतिष्ठा को संभावित नुकसान शामिल हैं।

\begin{itemize}
\tightlist
\item
  पूर्व-प्रशिक्षण डेटा किसी कार्य या डेटासेट के आधार पर किसी मॉडल को प्रशिक्षित करने
  को बातता है।
\item
  फ़ाइन-ट्यूनिंग में पहले से ही प्रशिक्षित मॉडल को, ओर अधिक संग्रहहीत एवं चयनित डेटासेट
  से एक संकीर्ण विषय या अधिक केंद्रित लक्ष्य ढालना शामिल है। इस डेटासेट मे इनपुट एवं
  उससे संबंधित वांछित आउटपुट के उदाहरण होते हैं।
\item
  एम्बेडिंग प्रक्रिया मे श्रेणीबद्ध डेटा (categorical data) को संख्यात्मक प्रतिनिधित्व
  (numerical representation) मे ढालना शामिल है, जिससे भाषा मॉडल को प्रशिक्षित
  कर सकते है। इसमे टेक्स्ट डेटा के शब्दों या वाक्यांशों को एक सतत वेक्टर स्पेस में वैक्टर में
  प्रस्तुत करते है। वेक्टर आमतौर पर टेक्स्ट डेटा को टेक्स्ट के बड़े संग्रह पर प्रशिक्षित न्युरल
  नेटवर्क (neural network) में देकरा मिलता है।
\end{itemize}

डेटा विषाक्तता को एक अखंडता हमला है क्योंकि प्रशिक्षण डेटा के साथ छेड़छाड़ से मॉडल की
सही भविष्यवाणियां करने की क्षमता प्रभावित होती है। स्वाभाविक रूप से, बाहरी डेटा
स्रोत उच्च जोखिम पेश करते हैं क्योंकि मॉडल निर्माताओं के पास डेटा पर नियंत्रण एवं उच्च
स्तर का विश्वास नहीं होता है कि, सामग्री में पक्षपातपूर्ण, गलत तथा अनुचित अनुचित
जनकारिया तो नहीं है।

\subsubsection{कमज़ोरी के सामान्य
उदाहरण}\label{ux915ux92eux95bux930-ux915-ux938ux92eux928ux92f-ux909ux926ux939ux930ux923}

\begin{enumerate}
\def\labelenumi{\arabic{enumi}.}
\tightlist
\item
  एक दुर्भावनापूर्ण व्यक्ति ,या एक प्रतियोगी ब्रांड जानबूझकर गलत या दुर्भावनापूर्ण
  दस्तावेज़ बनाता है, जो किसी मॉडल के प्रशिक्षण डेटा पर लक्षित होते हैं।
\item
  पीड़ित मॉडल गलत जानकारी का इस्तेमाल करके ट्रेनिंग होता है, जो उसके यूजर के जेनेरेटिव
  AI प्रॉम्प्ट के आउटपुट में दिखाई देती है।
\item
  किसी मॉडल को ऐसे डेटा का इस्तेमाल करके प्रशिक्षित किया गया है, जिस डेटा के स्रोत,
  उत्पत्ति या सामग्री की पुष्टि नहीं की गई है।
\item
  बुनियादी ढांचे के भीतर स्थित मॉडल के पास प्रशिक्षण डेटा के लिए अप्रतिबंधित पहुंच तथा
  अपर्याप्त सैंडबॉक्सिंग होती है। इससे जनरेटिव AI संकेतों के आउटपुट पर नकारात्मक प्रभाव
  पड़ता है, और प्रबंधन की दृष्टि से नियंत्रण की हानि होती है।
\end{enumerate}

चाहे LLM का डेवलपर, ग्राहक या सामान्य यूजर हो, यह समझना महत्वपूर्ण है कि
गैर-मालिकाना LLM का प्रयोग करते समय किस प्रकार सुरखा सम्बन्धी कमजोरियां आपके LLM
एप्लिकेशन के भीतर जोखिमों को उत्पन्न करती है।

\subsubsection{आक्रमण के
उदाहरण}\label{ux906ux915ux930ux92eux923-ux915-ux909ux926ux939ux930ux923}

\begin{enumerate}
\def\labelenumi{\arabic{enumi}.}
\tightlist
\item
  LLM जेनरेटिव AI प्रॉम्प्ट के आउटपुट एप्लिकेशन के यूज़रओं को गुमराह कर सकता है जिससे
  पक्षपात तथा अनुचर पूर्ण राय या इससे भी खराब ,घृणा अपराध आदि हो सकते हैं।
\item
  यदि प्रशिक्षण डेटा को सही ढंग से साफ़ नहीं किया गया है, तो एप्लिकेशन का कोई
  दुर्भावनापूर्ण यूजर मॉडल को प्रभावित या उसमे विषाक्त डेटा डालने का प्रयास कर सकता
  है । जिससे की मॉडल पक्षपाती और झूठे डेटा को अपना ले ।
\item
  एक दुर्भावनापूर्ण व्यक्ति या प्रतियोगी जानबूझकर गलत दस्तावेज़ बनाता है जो एक मॉडल
  के प्रशिक्षण डेटा पर लक्षित होते हैं। पीड़ित मॉडल इस झूठी जानकारी का उपयोग करके
  प्रशिक्षण लेता है जो उसके यूजर को जेनरेटिव AI संकेतों के आउटपुट में दिखाई पड़ता है।
\item
  यदि मॉडल को प्रशिक्षित करने के लिए LLM एप्लिकेशन इनपुट के क्लाइंट का उपयोग किया
  जाता है तो अपर्याप्त स्वच्छता और फ़िल्टरिंग से प्रॉम्प्ट इंजेक्शन आक्रमण वेक्टर हो सकता
  है। यानी गलत डेटा किसी क्लाइंट से प्रॉम्प्ट इंजेक्शन के रूप में मॉडल में इनपुट किया जाता
  है, तो इसे स्वाभाविक रूप से मॉडल डेटा में चित्रित किया जा सकता है।
\end{enumerate}

\subsubsection{बचाव कैसे करें}\label{ux92cux91aux935-ux915ux938-ux915ux930}

\begin{enumerate}
\def\labelenumi{\arabic{enumi}.}
\tightlist
\item
  प्रशिक्षण डेटा की आपूर्ति श्रृंखला को सत्यापित करें, खासकर जब बाहरी स्रोत से प्राप्त
  किया गया हो और साथ ही ``SBOM'' (सॉफ़्टवेयर सामग्री का बिल) पद्धति के समान
  सत्यापन भी बनाए रखा जाए।
\item
  प्रशिक्षण और फाइन-ट्यूनिंग दोनों चरणों के दौरान प्राप्त लक्षित डेटा स्रोतों और डेटा
  की सही वैधता को सत्यापित करें।
\item
  सबसे पहले LLM के उपयोग और उसकी ऐप्लिकेशन की पुष्टि करें। अलग-अलग प्रशिक्षण डेटा के
  ज़रिये मॉडल तैयार करें या फ़ाइन-ट्यूनिंग करें, ताकि इसके निर्धारित यूज़-केस के अनुसार
  ज़्यादा बारीक और सटीक जनरेटिव AI आउटपुट तैयार किये जा सके।
\item
  सुनिश्चित करें कि मॉडल को अनपेक्षित डेटा स्रोतों को स्क्रैप करने से रोकने के लिए
  पर्याप्त सैंडबॉक्सिंग मौजूद हो, जो मशीन लर्निंग आउटपुट में बाधा डाल सकती है।
\item
  ग़लत डेटा का वॉल्यूम नियंत्रित करने के लिए, प्रशिक्षण डेटा या डेटा स्रोतों की श्रेणियों
  के लिए सख्त इनपुट फ़िल्टर का इस्तेमाल करें। डेटा सैनिटाइज़ेशन, सांख्यिकीय आउटलेयर और
  विसंगति का पता लगाने की तकनीकों के साथ फाइन-ट्यूनिंग प्रक्रिया में प्रतिकूल डेटा का
  पता लगाया जा सके और उसे संभावित रूप से फीड होने से बचाया जा सके।
\item
  प्रतिकूल मजबूती तकनीकें जैसे कि फ़ेडरेटेड लर्निंग (federated learning) और आउटलायर के
  प्रभाव को कम करने के लिये प्रतिबंध या प्रशिक्षण डेटा में गड़बड़ी से बचने के लिए प्रतिकूल
  प्रशिक्षण।
\item
  ``MLSecOps'' का तरीका यह भी हो सकता है कि ऑटो पॉइजनिंग तकनीक की मदद से
  प्रशिक्षण जीवनचक्र में प्रतिकूल मजबूती को शामिल किया जाए।
\item
  इसका एक उदाहरण ऑटोपॉइज़न परीक्षण है , जिसमें कॉन्टेंट इंजेक्शन अटैक (``LLM
  प्रतिक्रियाओं में अपने ब्रांड को इंजेक्ट कैसे करें'') और रिफ़्यूज़ल अटैक (``मॉडल को जवाब
  देने से मना करना'') जैसे हमले शामिल हैं, जिन्हें इस तरीके से पूरा किया जा सकता है।
\item
  प्रशिक्षण चरण के दौरान नुकसान को मापकर और विशिष्ट परीक्षण इनपुट पर मॉडल
  व्यवहार का विश्लेषण करके विषाक्तता का पता लगाने के लिए प्रशिक्षित मॉडल का
  विश्लेषण करना।
\item
  एक सीमा से अधिक विषम प्रतिक्रियाओं की संख्या पर निगरानी रखना और सचेत करना।
\item
  प्रतिक्रियाओं और ऑडिटिंग की समीक्षा के लिए मानव का इस्तेमाल।
\item
  अनचाहे परिणामों के खिलाफ बेंचमार्क करने के लिए समर्पित LLM लागू करें और रीनफोर्समेंट
  लर्निंग तकनीकों का उपयोग करके अन्य LLM को प्रशिक्षित करें।
\item
  LLM जीवनचक्र के परीक्षण चरणों में LLM-आधारित रेड टीम अभ्यास या LLM की कमजोरियों
  को ढूंढे
\end{enumerate}

\subsubsection{संदर्भ लिंक}\label{ux938ux926ux930ux92d-ux932ux915}

\begin{enumerate}
\def\labelenumi{\arabic{enumi}.}
\tightlist
\item
  \href{https://stanford-cs324.github.io/winter2022/lectures/data/}{Stanford
  Research Paper}:CS324: Stanford Research
\item
  \href{https://www.csoonline.com/article/3613932/how-data-poisoning-attacks-corrupt-machine-learning-models.html}{How
  data poisoning attacks corrupt machine learning models}: CSO Online
\item
  \href{https://atlas.mitre.org/studies/AML.CS0009/}{MITRE ATLAS
  (framework) Tay Poisoning}: MITRE ATLAS
\item
  \href{https://blog.mithrilsecurity.io/poisongpt-how-we-hid-a-lobotomized-llm-on-hugging-face-to-spread-fake-news/}{PoisonGPT:
  How we hid a lobotomized LLM on Hugging Face to spread fake news}:
  Mithril Security
\item
  \href{https://kai-greshake.de/posts/inject-my-pdf/}{Inject My PDF:
  Prompt Injection for your Resume}: Kai Greshake
\item
  \href{https://towardsdatascience.com/backdoor-attacks-on-language-models-can-we-trust-our-models-weights-73108f9dcb1f}{Backdoor
  Attacks on Language Models}: Towards Data Science
\item
  \href{https://arxiv.org/abs/2305.00944}{Poisoning Language Models
  During Instruction}: Arxiv White Paper
\item
  \href{https://arxiv.org/abs/2306.04959}{FedMLSecurity:arXiv:2306.04959}:
  Arxiv White Paper
\item
  \href{https://softwarecrisis.dev/letters/the-poisoning-of-chatgpt/}{The
  poisoning of ChatGPT}: Software Crisis Blog
\item
  \href{https://www.youtube.com/watch?v=h9jf1ikcGyk}{Poisoning Web-Scale
  Training Datasets - Nicholas Carlini \textbar{} Stanford MLSys \#75}:
  YouTube Video
\item
  \href{https://cyclonedx.org/capabilities/mlbom/}{OWASP CycloneDX
  v1.5}: OWASP CycloneDX
\end{enumerate}

\end{document}
