% Options for packages loaded elsewhere
\PassOptionsToPackage{unicode}{hyperref}
\PassOptionsToPackage{hyphens}{url}
%
\documentclass[
]{article}
\usepackage{amsmath,amssymb}
\usepackage{iftex}
\ifPDFTeX
  \usepackage[T1]{fontenc}
  \usepackage[utf8]{inputenc}
  \usepackage{textcomp} % provide euro and other symbols
\else % if luatex or xetex
  \usepackage{unicode-math} % this also loads fontspec
  \defaultfontfeatures{Scale=MatchLowercase}
  \defaultfontfeatures[\rmfamily]{Ligatures=TeX,Scale=1}
\fi
\usepackage{lmodern}
\ifPDFTeX\else
  % xetex/luatex font selection
\fi
% Use upquote if available, for straight quotes in verbatim environments
\IfFileExists{upquote.sty}{\usepackage{upquote}}{}
\IfFileExists{microtype.sty}{% use microtype if available
  \usepackage[]{microtype}
  \UseMicrotypeSet[protrusion]{basicmath} % disable protrusion for tt fonts
}{}
\makeatletter
\@ifundefined{KOMAClassName}{% if non-KOMA class
  \IfFileExists{parskip.sty}{%
    \usepackage{parskip}
  }{% else
    \setlength{\parindent}{0pt}
    \setlength{\parskip}{6pt plus 2pt minus 1pt}}
}{% if KOMA class
  \KOMAoptions{parskip=half}}
\makeatother
\usepackage{xcolor}
\setlength{\emergencystretch}{3em} % prevent overfull lines
\providecommand{\tightlist}{%
  \setlength{\itemsep}{0pt}\setlength{\parskip}{0pt}}
\setcounter{secnumdepth}{-\maxdimen} % remove section numbering
\usepackage{bookmark}
\IfFileExists{xurl.sty}{\usepackage{xurl}}{} % add URL line breaks if available
\urlstyle{same}
\hypersetup{
  hidelinks,
  pdfcreator={LaTeX via pandoc}}

\author{}
\date{}

\begin{document}

\subsection{LLM07: Unsicheres
Plug-in-Design}\label{llm07-unsicheres-plug-in-design}

\subsubsection{Beschreibung}\label{beschreibung}

LLM-Plug-ins sind Erweiterungen, die, wenn sie aktiviert sind, bei
Benutzerinteraktionen automatisch vom Modell aufgerufen werden. Sie
werden von der Modell-Integrationsplattform gesteuert, und die Anwendung
hat möglicherweise keine Kontrolle über ihre Ausführung, insbesondere
wenn das Modell von einer anderen Instanz gehostet wird. Weiterhin ist
es wahrscheinlich, dass Plug-ins Freitexteingaben aus dem Modell ohne
Validierung oder Typüberprüfung implementieren, um Beschränkungen der
Kontextgröße zu umgehen. Dadurch können Angreifende eine böswillige
Anfrage an das Plug-in stellen, was zu einer Vielzahl von unerwünschtem
Verhalten bis hin zur Remote-Code-Ausführung führen kann.

Der Schaden, der durch böswillige Eingaben verursacht wird, hängt oft
von unzureichenden Zugriffskontrollen und dem Versäumnis ab, die
Autorisierung über Plug-ins hinweg zu verfolgen. Unzureichende
Zugriffskontrollen ermöglichen es einem Plug-in, anderen Plug-ins blind
zu vertrauen und davon auszugehen, dass die Eingaben von einem Menschen
stammen. Solche unzureichenden Zugriffskontrollen können dazu führen,
dass böswillige Eingaben schädliche Folgen haben, von der
Datenexfiltration über die Ausführung von Remote-Code bis hin zur
Privilegieneskalation.

Dieser Abschnitt konzentriert sich auf die Erstellung von LLM-Plug-ins
und nicht auf Plug-ins von Drittanbietern, die durch
LLM-Supply-Chain-Schwachstellen abgedeckt werden.

\subsubsection{Gängige Beispiele für
Schwachstellen}\label{guxe4ngige-beispiele-fuxfcr-schwachstellen}

\begin{enumerate}
\def\labelenumi{\arabic{enumi}.}
\tightlist
\item
  Ein Plug-in akzeptiert alle Parameter in einem einzigen Textfeld
  anstelle von eigenständig eingegebenen Parametern.
\item
  Ein Plug-in akzeptiert Konfigurationsstrings anstelle von Parametern,
  die alle Konfigurationseinstellungen überschreiben können.
\item
  Ein Plug-in akzeptiert direkt SQL- oder Programmieranweisungen
  anstelle von Parametern.
\item
  Die Authentifizierung erfolgt ohne explizite Autorisierung für ein
  bestimmtes Plug-in.
\item
  Ein Plug-in behandelt alle LLM-Inhalte so, als ob sie vollständig vom
  Menschen erstellt wurden, und führt jede angeforderte Aktion ohne
  zusätzliche Autorisierung aus.
\end{enumerate}

\subsubsection{Präventions- und
Mitigationsstrategien}\label{pruxe4ventions--und-mitigationsstrategien}

\begin{enumerate}
\def\labelenumi{\arabic{enumi}.}
\tightlist
\item
  Plug-ins sollten, wo immer möglich, streng parametrisierte Eingaben
  erzwingen und Typ- und Bereichsprüfungen für Eingaben vorsehen. Wenn
  dies nicht möglich ist, sollte eine zweite Schicht von typisierten
  Aufrufen eingeführt werden, die die Anfragen parst und eine
  Validierung und Bereinigung durchführt. Wenn Freitexteingaben aufgrund
  der Anwendungssemantik akzeptiert werden müssen, sollten diese
  sorgfältig geprüft werden, um sicherzustellen, dass keine potenziell
  schädlichen Methoden aufgerufen werden.
\item
  Plug-in-Entwickelnde sollten die Empfehlungen aus dem OWASP ASVS
  (Application Security Verification Standard) anwenden, um eine
  angemessene Validierung und Bereinigung von Eingaben sicherzustellen.
\item
  Plug-ins sollten gründlich geprüft und getestet werden, um eine
  angemessene Validierung zu gewährleisten. Verwenden Sie statische
  Anwendungssicherheitstests (SAST) sowie dynamische und interaktive
  Anwendungstests (DAST, IAST) in den Entwicklungspipelines.
\item
  Plug-ins sollten so entworfen werden, dass die Auswirkungen der
  Ausnutzung unsicherer Eingabeparameter gemäß den OWASP ASVS Access
  Control Guidelines minimiert werden. Dies beinhaltet eine
  Zugriffskontrolle mit den geringsten Rechten, die so wenig
  Funktionalität wie möglich preisgibt, aber dennoch die gewünschte
  Funktion erfüllt.
\item
  Plug-ins sollten geeignete Authentifizierungsidentitäten wie OAuth2
  verwenden, um eine effektive Autorisierung und Zugriffskontrolle
  anzuwenden. Überdies sollten API-Schlüssel verwendet werden, um den
  Kontext für benutzerdefinierte Autorisierungsentscheidungen
  bereitzustellen, die den Pfad des Plug-ins und nicht den interaktiven
  Standardbenutzer widerspiegeln.
\item
  Verlangen Sie eine manuelle Benutzerautorisierung und Bestätigung
  aller von vertraulichen Plug-ins durchgeführten Aktionen.
\item
  Plug-ins sind in der Regel REST-APIs. Daher sollten Entwickelnde die
  Empfehlungen in OWASP Top 10 API Security Risks - 2023 befolgen, um
  allgemeine Schwachstellen zu minimieren.
\end{enumerate}

\subsubsection{Beispiele für
Angriffsszenarien}\label{beispiele-fuxfcr-angriffsszenarien}

\begin{enumerate}
\def\labelenumi{\arabic{enumi}.}
\tightlist
\item
  Ein Plug-in akzeptiert eine Basis-URL und weist das LLM an, die URL
  mit einer Anfrage zu kombinieren, um Wettervorhersagen zu erhalten,
  die in die Bearbeitung der Benutzeranfrage einfließen. Böswillige
  Personen können eine Anfrage so gestalten, dass die URL auf eine von
  ihnen kontrollierte Domäne verweist, wodurch sie ihre eigenen Inhalte
  über diese in das LLM-System einspeisen können.
\item
  Ein Plug-in akzeptiert eine freie Eingabe in ein einzelnes Feld, die
  nicht validiert wird. Angreifende liefern sorgfältig gestaltete
  Payloads, um Fehlermeldungen auszuspähen. Anschließend nutzen sie
  bekannte Sicherheitslücken von Drittanbietern aus, um Code
  auszuführen, Daten zu exfiltrieren oder Rechte zu erweitern.
\item
  Ein Plug-in, das zum Abrufen von Embeddings aus einem Vektorspeicher
  verwendet wird, akzeptiert Konfigurationsparameter als
  Verbindungsstring ohne jegliche Validierung. Dadurch können
  Angreifende experimentieren und auf andere Vektorspeicher zugreifen,
  indem sie Namen oder Host-Parameter ändern und Embeddings
  exfiltrieren, auf die sie keinen Zugriff haben sollten.
\item
  Ein Plug-in akzeptiert SQL WHERE-Klauseln als erweiterte Filter, die
  dann an die SQL-Bedingungen angehängt werden. Dadurch können
  Angreifende einen SQL-Angriff durchführen.
\item
  Angreifende nutzen eine indirekte Prompt Injection aus, um ein
  unsicheres Codeverwaltungs-Plug-in ohne Eingabevalidierung und mit
  schwacher Zugriffskontrolle auszunutzen, um den Besitz von Repositorys
  zu übertragen und Personen von ihren Repositorys auszuschließen.
\end{enumerate}

\subsubsection{Referenzen}\label{referenzen}

\begin{enumerate}
\def\labelenumi{\arabic{enumi}.}
\tightlist
\item
  \href{https://platform.openai.com/docs/plugins/introduction}{OpenAI
  ChatGPT Plugins}: \textbf{ChatGPT Developer's Guide}
\item
  \href{https://platform.openai.com/docs/plugins/introduction/plugin-flow}{OpenAI
  ChatGPT Plugins - Plugin Flow}: \textbf{OpenAI Documentation}
\item
  \href{https://platform.openai.com/docs/plugins/authentication/service-level}{OpenAI
  ChatGPT Plugins - Authentication}: \textbf{OpenAI Documentation}
\item
  \href{https://github.com/openai/chatgpt-retrieval-plugin}{OpenAI
  Semantic Search Plugin Sample}: \textbf{OpenAI Github}
\item
  \href{https://embracethered.com/blog/posts/2023/chatgpt-plugin-vulns-chat-with-code/}{Plugin
  Vulnerabilities: Visit a Website and Have Your Source Code Stolen}:
  \textbf{Embrace The Red}
\item
  \href{https://embracethered.com/blog/posts/2023/chatgpt-cross-plugin-request-forgery-and-prompt-injection./}{ChatGPT
  Plugin Exploit Explained: From Prompt Injection to Accessing Private
  Data} \textbf{Embrace The Red}
\item
  \href{https://embracethered.com/blog/posts/2023/chatgpt-cross-plugin-request-forgery-and-prompt-injection./}{ChatGPT
  Plugin Exploit Explained: From Prompt Injection to Accessing Private
  Data}: \textbf{Embrace The Red}
\item
  \href{https://owasp-aasvs4.readthedocs.io/en/latest/V5.html\#validation-sanitization-and-encoding}{OWASP
  ASVS - 5 Validation, Sanitization and Encoding}: \textbf{OWASP AASVS}
\item
  \href{https://owasp-aasvs4.readthedocs.io/en/latest/V4.1.html\#general-access-control-design}{OWASP
  ASVS 4.1 General Access Control Design}: \textbf{OWASP AASVS}
\item
  \href{https://owasp.org/API-Security/editions/2023/en/0x11-t10/}{OWASP
  Top 10 API Security Risks -- 2023}: \textbf{OWASP}
\end{enumerate}

\end{document}
