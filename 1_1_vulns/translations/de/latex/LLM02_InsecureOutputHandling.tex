% Options for packages loaded elsewhere
\PassOptionsToPackage{unicode}{hyperref}
\PassOptionsToPackage{hyphens}{url}
%
\documentclass[
]{article}
\usepackage{amsmath,amssymb}
\usepackage{iftex}
\ifPDFTeX
  \usepackage[T1]{fontenc}
  \usepackage[utf8]{inputenc}
  \usepackage{textcomp} % provide euro and other symbols
\else % if luatex or xetex
  \usepackage{unicode-math} % this also loads fontspec
  \defaultfontfeatures{Scale=MatchLowercase}
  \defaultfontfeatures[\rmfamily]{Ligatures=TeX,Scale=1}
\fi
\usepackage{lmodern}
\ifPDFTeX\else
  % xetex/luatex font selection
\fi
% Use upquote if available, for straight quotes in verbatim environments
\IfFileExists{upquote.sty}{\usepackage{upquote}}{}
\IfFileExists{microtype.sty}{% use microtype if available
  \usepackage[]{microtype}
  \UseMicrotypeSet[protrusion]{basicmath} % disable protrusion for tt fonts
}{}
\makeatletter
\@ifundefined{KOMAClassName}{% if non-KOMA class
  \IfFileExists{parskip.sty}{%
    \usepackage{parskip}
  }{% else
    \setlength{\parindent}{0pt}
    \setlength{\parskip}{6pt plus 2pt minus 1pt}}
}{% if KOMA class
  \KOMAoptions{parskip=half}}
\makeatother
\usepackage{xcolor}
\setlength{\emergencystretch}{3em} % prevent overfull lines
\providecommand{\tightlist}{%
  \setlength{\itemsep}{0pt}\setlength{\parskip}{0pt}}
\setcounter{secnumdepth}{-\maxdimen} % remove section numbering
\usepackage{bookmark}
\IfFileExists{xurl.sty}{\usepackage{xurl}}{} % add URL line breaks if available
\urlstyle{same}
\hypersetup{
  hidelinks,
  pdfcreator={LaTeX via pandoc}}

\author{}
\date{}

\begin{document}

\subsection{LLM02: Unsichere
Ausgabeverarbeitung}\label{llm02-unsichere-ausgabeverarbeitung}

\subsubsection{Beschreibung}\label{beschreibung}

Unsichere Ausgabeverarbeitung bezieht sich speziell auf die
unzureichende Validierung, Bereinigung und Handhabung von Ausgaben, die
von Large Language Models erzeugt werden, bevor sie an andere
Komponenten und Systeme weitergeleitet werden. Da der von LLMs erzeugte
Inhalt durch Prompt-Eingaben gesteuert werden kann, ähnelt dieses
Verhalten dem indirekten Zugriff des Benutzers auf zusätzliche
Funktionen.

Unsichere Ausgabeverarbeitung unterscheidet sich von Übermäßiger
Abhängigkeit insofern, als sie sich mit den Ausgaben befasst, die von
LLMs generiert werden, bevor sie weitergeleitet werden. Im Gegensatz
dazu konzentriert sich Übermäßige Abhängigkeit auf allgemeinere Bedenken
hinsichtlich der Angewiesenheit auf die Genauigkeit und Angemessenheit
des LLM-Outputs.

Die erfolgreiche Ausnutzung einer Schwachstelle in der unsicheren
Ausgabeverarbeitung kann zu XSS (Cross-Site Scripting) und CSRF
(Cross-Site Request Forgery) in Webbrowsern sowie zu SSRF (Server-Side
Request Forgery), Rechteerweiterung (Privilege Escalation) oder
Remote-Code-Ausführung in Backend-Systemen führen.

Die folgenden Bedingungen können die Auswirkungen dieser Schwachstelle
verstärken:

\begin{itemize}
\tightlist
\item
  Die Anwendung gewährt dem LLM Privilegien, die über die für den
  Endbenutzer vorgesehenen Privilegien hinausgehen, was eine Eskalation
  der Privilegien oder die Ausführung von Remote-Code ermöglicht.
\item
  Die Anwendung ist anfällig für indirekte Prompt Injection-Angriffe,
  die es Angreifenden ermöglichen, privilegierten Zugriff auf die
  Umgebung eines Zielbenutzers zu erlangen.
\item
  Plug-ins von Drittanbietern validieren Eingaben nicht ausreichend.
\end{itemize}

\subsubsection{Gängige Beispiele für
Schwachstellen}\label{guxe4ngige-beispiele-fuxfcr-schwachstellen}

\begin{enumerate}
\def\labelenumi{\arabic{enumi}.}
\tightlist
\item
  Die Ausgabe des LLM wird direkt in eine Systemshell oder eine ähnliche
  Funktion wie exec oder eval eingegeben, was zu einer
  Remote-Code-Ausführung führt.
\item
  JavaScript oder Markdown wird vom LLM generiert und an die aufrufende
  Person zurückgegeben. Der Code wird dann vom Browser interpretiert,
  was zu XSS führt.
\end{enumerate}

\subsubsection{Präventions- und
Mitigationsstrategien}\label{pruxe4ventions--und-mitigationsstrategien}

\begin{enumerate}
\def\labelenumi{\arabic{enumi}.}
\tightlist
\item
  Behandeln Sie das Sprachmodell mit einem Zero-Trust-Ansatz und wenden
  Sie eine geeignete Eingabevalidierung auf die Antworten an, die vom
  Modell an die Backend-Funktionen gesendet werden.
\item
  Befolgen Sie die OWASP ASVS (Application Security Verification
  Standard) Richtlinien, um eine effektive Eingabevalidierung und
  -bereinigung zu gewährleisten.
\item
  Encoden Sie die Modellausgabe zurück an den Benutzer, um unerwünschte
  Codeausführung durch JavaScript oder Markdown zu verhindern. Der OWASP
  ASVS bietet detaillierte Anweisungen zum Output Encoding.
\end{enumerate}

\subsubsection{Beispiele für
Angriffsszenarien}\label{beispiele-fuxfcr-angriffsszenarien}

\begin{enumerate}
\def\labelenumi{\arabic{enumi}.}
\tightlist
\item
  Eine Anwendung verwendet ein LLM-Plug-in, um Antworten für eine
  Chatbot-Funktion zu generieren. Das Plug-in bietet auch eine Reihe von
  administrativen Funktionen, die einem anderen privilegierten LLM zur
  Verfügung stehen. Das allgemeine LLM sendet seine Antwort direkt an
  das Plug-in, ohne die Ausgabe ordnungsgemäß zu validieren, was dazu
  führt, dass das Plug-in für Wartungsarbeiten heruntergefahren wird.
\item
  Eine Person verwendet ein von einem LLM betriebenes Tool, das
  Webseiten zusammenfasst, um eine kurze Übersicht über einen Artikel zu
  erstellen. Die Website enthält eine Eingabeaufforderung, die das LLM
  anweist, sensible Inhalte entweder von der Website oder aus der
  Konversation des Benutzers zu erfassen. Anschließend kann der LLM die
  sensiblen Daten verschlüsseln und ohne Validierung oder Filterung der
  Ausgabe an einen von Angreifenden kontrollierten Server senden.
\item
  Ein LLM ermöglicht es Personen, SQL-Abfragen für eine
  Backend-Datenbank über eine Chat-ähnliche Funktion zu erstellen. Eine
  Person stellt eine Abfrage zum Löschen aller Datenbanktabellen. Wenn
  die vom LLM erstellte Abfrage nicht überprüft wird, könnten alle
  Datenbanktabellen gelöscht werden.
\item
  Eine Webanwendung verwendet einen LLM, um Inhalte aus Benutzereingaben
  zu generieren, ohne die Ausgabe zu bereinigen. Angreifende könnten
  eine manipulierte Anfrage einreichen, die das LLM dazu veranlasst,
  eine unbereinigte JavaScript-Payload zurückzugeben, die zu XSS führt,
  wenn sie im Browser des Opfers ausgeführt wird. Unzureichende
  Validierung von Anfragen ermöglicht diesen Angriff.
\end{enumerate}

\subsubsection{Referenzen}\label{referenzen}

\begin{enumerate}
\def\labelenumi{\arabic{enumi}.}
\tightlist
\item
  \href{https://security.snyk.io/vuln/SNYK-PYTHON-LANGCHAIN-5411357}{Arbitrary
  Code Execution}: \textbf{Snyk Security Blog}
\item
  \href{https://embracethered.com/blog/posts/2023/chatgpt-cross-plugin-request-forgery-and-prompt-injection./}{ChatGPT
  Plugin Exploit Explained: From Prompt Injection to Accessing Private
  Data}: \textbf{Embrace The Red}
\item
  \href{https://systemweakness.com/new-prompt-injection-attack-on-chatgpt-web-version-ef717492c5c2?gi=8daec85e2116}{New
  prompt injection attack on ChatGPT web version. Markdown images can
  steal your chat data.}: \textbf{System Weakness}
\item
  \href{https://embracethered.com/blog/posts/2023/ai-injections-threats-context-matters/}{Don't
  blindly trust LLM responses. Threats to chatbots}: \textbf{Embrace The
  Red}
\item
  \href{https://aivillage.org/large\%20language\%20models/threat-modeling-llm/}{Threat
  Modeling LLM Applications}: \textbf{AI Village}
\item
  \href{https://owasp-aasvs4.readthedocs.io/en/latest/V5.html\#validation-sanitization-and-encoding}{OWASP
  ASVS - 5 Validation, Sanitization and Encoding}: \textbf{OWASP AASVS}
\end{enumerate}

\end{document}
