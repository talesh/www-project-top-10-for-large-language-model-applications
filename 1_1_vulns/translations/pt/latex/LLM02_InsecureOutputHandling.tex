% Options for packages loaded elsewhere
\PassOptionsToPackage{unicode}{hyperref}
\PassOptionsToPackage{hyphens}{url}
%
\documentclass[
]{article}
\usepackage{amsmath,amssymb}
\usepackage{iftex}
\ifPDFTeX
  \usepackage[T1]{fontenc}
  \usepackage[utf8]{inputenc}
  \usepackage{textcomp} % provide euro and other symbols
\else % if luatex or xetex
  \usepackage{unicode-math} % this also loads fontspec
  \defaultfontfeatures{Scale=MatchLowercase}
  \defaultfontfeatures[\rmfamily]{Ligatures=TeX,Scale=1}
\fi
\usepackage{lmodern}
\ifPDFTeX\else
  % xetex/luatex font selection
\fi
% Use upquote if available, for straight quotes in verbatim environments
\IfFileExists{upquote.sty}{\usepackage{upquote}}{}
\IfFileExists{microtype.sty}{% use microtype if available
  \usepackage[]{microtype}
  \UseMicrotypeSet[protrusion]{basicmath} % disable protrusion for tt fonts
}{}
\makeatletter
\@ifundefined{KOMAClassName}{% if non-KOMA class
  \IfFileExists{parskip.sty}{%
    \usepackage{parskip}
  }{% else
    \setlength{\parindent}{0pt}
    \setlength{\parskip}{6pt plus 2pt minus 1pt}}
}{% if KOMA class
  \KOMAoptions{parskip=half}}
\makeatother
\usepackage{xcolor}
\setlength{\emergencystretch}{3em} % prevent overfull lines
\providecommand{\tightlist}{%
  \setlength{\itemsep}{0pt}\setlength{\parskip}{0pt}}
\setcounter{secnumdepth}{-\maxdimen} % remove section numbering
\usepackage{bookmark}
\IfFileExists{xurl.sty}{\usepackage{xurl}}{} % add URL line breaks if available
\urlstyle{same}
\hypersetup{
  hidelinks,
  pdfcreator={LaTeX via pandoc}}

\author{}
\date{}

\begin{document}

\subsection{LLM02: Manipulação Insegura de
Output}\label{llm02-manipulauxe7uxe3o-insegura-de-output}

\subsubsection{Descrição}\label{descriuxe7uxe3o}

A Manipulação Insegura de Output refere-se especificamente à validação,
sanitização e tratamento insuficientes das saídas geradas por modelos de
linguagem grandes (LLMs) antes de serem encaminhadas para outros
componentes e sistemas. Como o conteúdo gerado pelo LLM pode ser
controlado pela entrada do prompt, esse comportamento é semelhante a
fornecer aos usuários acesso indireto a funcionalidades adicionais.

A Manipulação Insegura de Output difere da Dependência Excessiva, pois
lida com as saídas geradas pelo LLM antes de serem encaminhadas,
enquanto a Dependência Excessiva se concentra em preocupações mais
amplas em torno da superdependência da precisão e adequação das saídas
do LLM.

A exploração bem-sucedida de uma vulnerabilidade de Manipulação Insegura
de Output pode resultar em XSS e CSRF em navegadores da web, bem como
SSRF, escalonamento de privilégios ou execução remota de código em
sistemas de backend.

As seguintes condições podem aumentar o impacto dessa vulnerabilidade:

\begin{itemize}
\tightlist
\item
  A aplicação concede privilégios ao LLM além do que é destinado aos
  usuários finais, possibilitando a escalada de privilégios ou a
  execução remota de código.
\item
  A aplicação é vulnerável a ataques de injeção de prompt indireta, o
  que pode permitir que um atacante ganhe acesso privilegiado ao
  ambiente de um usuário-alvo.
\item
  Plugins de terceiros que não validam adequadamente as entradas.
\end{itemize}

\subsubsection{Exemplos Comuns desta
Vulnerabilidade}\label{exemplos-comuns-desta-vulnerabilidade}

\begin{itemize}
\tightlist
\item
  A saída do LLM é inserida diretamente em um shell do sistema ou função
  semelhante, como exec ou eval, resultando na execução remota de
  código.
\item
  JavaScript ou Markdown é gerado pelo LLM e retornado a um usuário. O
  código é então interpretado pelo navegador, resultando em XSS.
\end{itemize}

\subsubsection{Estratégias de Prevenção e
Mitigação}\label{estratuxe9gias-de-prevenuxe7uxe3o-e-mitigauxe7uxe3o}

\begin{itemize}
\tightlist
\item
  Trate o modelo como qualquer outro usuário, adotando uma abordagem de
  confiança zero, e aplique validação adequada nas respostas
  provenientes do modelo para funções de backend.
\item
  Siga as diretrizes do OWASP ASVS (Application Security Verification
  Standard) para garantir uma validação e sanitização eficazes de
  entrada.
\item
  Codifique a saída do modelo de volta para os usuários para mitigar a
  execução indesejada de código JavaScript ou Markdown. O OWASP ASVS
  fornece orientações detalhadas sobre codificação de saída.
\end{itemize}

\subsubsection{Exemplos de Cenários de
Ataque}\label{exemplos-de-cenuxe1rios-de-ataque}

\begin{enumerate}
\def\labelenumi{\arabic{enumi}.}
\tightlist
\item
  Uma aplicação utiliza um plugin de LLM para gerar respostas para uma
  funcionalidade de chatbot. O plugin também oferece diversas funções
  administrativas acessíveis a outro LLM privilegiado. O LLM de
  propósito geral passa diretamente sua resposta, sem uma validação
  adequada de saída, para o plugin, causando o desligamento do plugin
  para manutenção.
\item
  Um usuário utiliza uma ferramenta de resumo de site alimentada por um
  LLM para gerar um resumo conciso de um artigo. O site inclui uma
  injeção de prompt instruindo o LLM a capturar conteúdo sensível do
  site ou da conversa do usuário. A partir daí, o LLM pode codificar os
  dados sensíveis e enviá-los, sem validação ou filtragem adequada de
  saída, para um servidor controlado pelo atacante.
\item
  Um LLM permite que os usuários criem consultas SQL para um banco de
  dados de backend por meio de uma funcionalidade semelhante a um chat.
  Um usuário solicita uma consulta para excluir todas as tabelas do
  banco de dados. Se a consulta elaborada pelo LLM não for examinada
  cuidadosamente, todas as tabelas do banco de dados serão excluídas.
\item
  Um aplicativo da web usa um LLM para gerar conteúdo a partir de
  prompts de texto do usuário sem sanitização de saída. Um atacante pode
  enviar um prompt manipulado fazendo com que o LLM retorne uma carga
  útil de JavaScript não sanitizada, resultando em XSS quando
  renderizado no navegador da vítima. A validação insuficiente dos
  prompts possibilitou esse ataque.
\end{enumerate}

\subsubsection{Links de Referência}\label{links-de-referuxeancia}

\begin{enumerate}
\def\labelenumi{\arabic{enumi}.}
\tightlist
\item
  \href{https://security.snyk.io/vuln/SNYK-PYTHON-LANGCHAIN-5411357}{Arbitrary
  Code Execution}: \textbf{Snyk Security Blog}
\item
  \href{https://embracethered.com/blog/posts/2023/chatgpt-cross-plugin-request-forgery-and-prompt-injection./}{ChatGPT
  Plugin Exploit Explained: From Prompt Injection to Accessing Private
  Data}: \textbf{Embrace The Red}
\item
  \href{https://systemweakness.com/new-prompt-injection-attack-on-chatgpt-web-version-ef717492c5c2?gi=8daec85e2116}{New
  prompt injection attack on ChatGPT web version. Markdown images can
  steal your chat data.}: \textbf{System Weakness}
\item
  \href{https://embracethered.com/blog/posts/2023/ai-injections-threats-context-matters/}{Don't
  blindly trust LLM responses. Threats to chatbots}: \textbf{Embrace The
  Red}
\item
  \href{https://aivillage.org/large\%20language\%20models/threat-modeling-llm/}{Threat
  Modeling LLM Applications}: \textbf{AI Village}
\item
  \href{https://owasp-aasvs4.readthedocs.io/en/latest/V5.html\#validation-sanitization-and-encoding}{OWASP
  ASVS - 5 Validation, Sanitization and Encoding}: \textbf{OWASP AASVS}
\end{enumerate}

\end{document}
