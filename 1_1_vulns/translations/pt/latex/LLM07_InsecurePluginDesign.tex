% Options for packages loaded elsewhere
\PassOptionsToPackage{unicode}{hyperref}
\PassOptionsToPackage{hyphens}{url}
%
\documentclass[
]{article}
\usepackage{amsmath,amssymb}
\usepackage{iftex}
\ifPDFTeX
  \usepackage[T1]{fontenc}
  \usepackage[utf8]{inputenc}
  \usepackage{textcomp} % provide euro and other symbols
\else % if luatex or xetex
  \usepackage{unicode-math} % this also loads fontspec
  \defaultfontfeatures{Scale=MatchLowercase}
  \defaultfontfeatures[\rmfamily]{Ligatures=TeX,Scale=1}
\fi
\usepackage{lmodern}
\ifPDFTeX\else
  % xetex/luatex font selection
\fi
% Use upquote if available, for straight quotes in verbatim environments
\IfFileExists{upquote.sty}{\usepackage{upquote}}{}
\IfFileExists{microtype.sty}{% use microtype if available
  \usepackage[]{microtype}
  \UseMicrotypeSet[protrusion]{basicmath} % disable protrusion for tt fonts
}{}
\makeatletter
\@ifundefined{KOMAClassName}{% if non-KOMA class
  \IfFileExists{parskip.sty}{%
    \usepackage{parskip}
  }{% else
    \setlength{\parindent}{0pt}
    \setlength{\parskip}{6pt plus 2pt minus 1pt}}
}{% if KOMA class
  \KOMAoptions{parskip=half}}
\makeatother
\usepackage{xcolor}
\setlength{\emergencystretch}{3em} % prevent overfull lines
\providecommand{\tightlist}{%
  \setlength{\itemsep}{0pt}\setlength{\parskip}{0pt}}
\setcounter{secnumdepth}{-\maxdimen} % remove section numbering
\usepackage{bookmark}
\IfFileExists{xurl.sty}{\usepackage{xurl}}{} % add URL line breaks if available
\urlstyle{same}
\hypersetup{
  hidelinks,
  pdfcreator={LaTeX via pandoc}}

\author{}
\date{}

\begin{document}

\subsection{LLM07: Design Inseguro de
Plugins}\label{llm07-design-inseguro-de-plugins}

\subsubsection{Descrição}\label{descriuxe7uxe3o}

Os plugins LLM são extensões que, quando habilitadas, são chamadas
automaticamente pelo modelo durante as interações do usuário. A
plataforma de integração do modelo os controla, e a aplicação pode não
ter controle sobre a execução, especialmente quando o modelo é hospedado
por outra parte. Além disso, os plugins provavelmente implementam
entradas de texto livre do modelo sem validação ou verificação de tipo
para lidar com limitações de tamanho de contexto. Isso permite que um
potencial atacante construa uma solicitação maliciosa para o plugin, o
que pode resultar em uma ampla gama de comportamentos indesejados,
incluindo execução remota de código.

O dano de inputs maliciosos muitas vezes depende de controles de acesso
insuficientes e da falha no rastreamento de autorização em plugins. O
controle de acesso inadequado permite que um plugin confie cegamente em
outros plugins e assuma que o usuário final forneceu as entradas. Esse
controle de acesso inadequado pode permitir que inputs maliciosos tenham
consequências prejudiciais que vão desde exfiltração de dados, execução
remota de código até escalonamento de privilégios.

Este item foca na criação de plugins LLM, em vez de plugins de
terceiros, que são abordados nas Vulnerabilidades de Cadeia de
Suprimentos LLM.

\subsubsection{Exemplos Comuns de
Vulnerabilidade}\label{exemplos-comuns-de-vulnerabilidade}

\begin{enumerate}
\def\labelenumi{\arabic{enumi}.}
\tightlist
\item
  Um plugin aceita todos os parâmetros em um único campo de texto em vez
  de parâmetros de entrada distintos.
\item
  Um plugin aceita strings de configuração em vez de parâmetros que
  podem substituir configurações inteiras.
\item
  Um plugin aceita instruções SQL ou de programação em bruto em vez de
  parâmetros.
\item
  A autenticação é realizada sem autorização explícita para um plugin
  específico.
\item
  Um plugin trata todo o conteúdo do LLM como se fosse criado
  inteiramente pelo usuário e executa quaisquer ações solicitadas sem
  exigir autorização adicional.
\end{enumerate}

\subsubsection{Estratégias de Prevenção e
Mitigação}\label{estratuxe9gias-de-prevenuxe7uxe3o-e-mitigauxe7uxe3o}

\begin{enumerate}
\def\labelenumi{\arabic{enumi}.}
\tightlist
\item
  Os plugins devem impor uma entrada estritamente parametrizada sempre
  que possível e incluir verificações de tipo e intervalo nas entradas.
  Quando isso não for possível, uma segunda camada de chamadas tipadas
  deve ser introduzida, analisando solicitações e aplicando validação e
  sanitização. Quando a entrada de formulário livre deve ser aceita
  devido à semântica da aplicação, ela deve ser cuidadosamente
  inspecionada para garantir que nenhum método potencialmente
  prejudicial esteja sendo chamado.
\item
  Os desenvolvedores de plugins devem aplicar as recomendações da OWASP
  no ASVS (Application Security Verification Standard) para garantir
  validação e sanitização adequadas de entrada.
\item
  Os plugins devem ser inspecionados e testados minuciosamente para
  garantir validação adequada. Use varreduras de Teste Estático de
  Segurança de Aplicações (SAST) e Teste Dinâmico e Interativo de
  Aplicações (DAST, IAST) nos pipelines de desenvolvimento.
\item
  Os plugins devem ser projetados para minimizar o impacto de qualquer
  exploração de parâmetro de entrada insegura seguindo as Diretrizes de
  Controle de Acesso da OWASP ASVS. Isso inclui controle de acesso de
  menor privilégio, expondo o mínimo de funcionalidade possível enquanto
  ainda realiza a função desejada.
\item
  Os plugins devem usar identidades de autenticação apropriadas, como
  OAuth2, para aplicar autorização e controle de acesso eficazes. Além
  disso, as Chaves de API devem ser usadas para fornecer contexto para
  decisões de autorização personalizadas que reflitam a rota do plugin,
  em vez do usuário interativo padrão.
\item
  Exigir autorização e confirmação manual do usuário para qualquer ação
  realizada por plugins sensíveis.
\item
  Os plugins são, tipicamente, APIs REST, então os desenvolvedores devem
  aplicar as recomendações encontradas nas Principais 10
  Vulnerabilidades de Segurança em APIs da OWASP - 2023 para minimizar
  vulnerabilidades genéricas.
\end{enumerate}

\subsubsection{Cenários de Ataque
Exemplo}\label{cenuxe1rios-de-ataque-exemplo}

\begin{enumerate}
\def\labelenumi{\arabic{enumi}.}
\tightlist
\item
  Um plugin aceita uma URL base e instrui o LLM a combinar a URL com uma
  consulta para obter previsões do tempo, que são incluídas no
  tratamento da solicitação do usuário. Um usuário mal-intencionado pode
  criar uma solicitação para que a URL aponte para um domínio que eles
  controlam, permitindo que eles injetem seu próprio conteúdo no sistema
  LLM por meio de seu domínio.
\item
  Um plugin aceita uma entrada de formulário livre em um único campo que
  não valida. Um atacante fornece cargas cuidadosamente elaboradas para
  realizar reconhecimento a partir de mensagens de erro. Em seguida, ele
  explora vulnerabilidades conhecidas de terceiros para executar código
  e realizar exfiltração de dados ou escalonamento de privilégios.
\item
  Um plugin usado para recuperar embeddings de um repositório de vetores
  aceita parâmetros de configuração como uma string de conexão sem
  validação. Isso permite que um atacante experimente e acesse outros
  repositórios de vetores alterando nomes ou parâmetros de host e
  exfiltra embeddings aos quais não deveria ter acesso.
\item
  Um plugin aceita cláusulas WHERE SQL como filtros avançados, que são
  então anexadas ao SQL de filtragem. Isso permite que um atacante
  execute um ataque SQL.
\item
  Um atacante usa injeção de prompt indireto para explorar um plugin de
  gerenciamento de código inseguro, sem validação de entrada e controle
  de acesso fraco, para transferir a propriedade do repositório e
  bloquear o usuário de seus repositórios.
\end{enumerate}

\subsubsection{Links de Referência}\label{links-de-referuxeancia}

\begin{enumerate}
\def\labelenumi{\arabic{enumi}.}
\tightlist
\item
  \href{https://platform.openai.com/docs/plugins/introduction}{OpenAI
  ChatGPT Plugins}: \textbf{ChatGPT Developer's Guide}
\item
  \href{https://platform.openai.com/docs/plugins/introduction/plugin-flow}{OpenAI
  ChatGPT Plugins - Plugin Flow}: \textbf{OpenAI Documentation}
\item
  \href{https://platform.openai.com/docs/plugins/authentication/service-level}{OpenAI
  ChatGPT Plugins - Authentication}: \textbf{OpenAI Documentation}
\item
  \href{https://github.com/openai/chatgpt-retrieval-plugin}{OpenAI
  Semantic Search Plugin Sample}: \textbf{OpenAI Github}
\item
  \href{https://embracethered.com/blog/posts/2023/chatgpt-plugin-vulns-chat-with-code/}{Plugin
  Vulnerabilities: Visit a Website and Have Your Source Code Stolen}:
  \textbf{Embrace The Red}
\item
  \href{https://embracethered.com/blog/posts/2023/chatgpt-cross-plugin-request-forgery-and-prompt-injection./}{ChatGPT
  Plugin Exploit Explained: From Prompt Injection to Accessing Private
  Data} \textbf{Embrace The Red}
\item
  \href{https://owasp-aasvs4.readthedocs.io/en/latest/V5.html\#validation-sanitization-and-encoding}{OWASP
  ASVS - 5 Validation, Sanitization and Encoding}: \textbf{OWASP AASVS}
\item
  \href{https://owasp-aasvs4.readthedocs.io/en/latest/V4.1.html\#general-access-control-design}{OWASP
  ASVS 4.1 General Access Control Design}: \textbf{OWASP AASVS}
\item
  \href{https://owasp.org/API-Security/editions/2023/en/0x11-t10/}{OWASP
  Top 10 API Security Risks -- 2023}: \textbf{OWASP}
\end{enumerate}

\end{document}
