% Options for packages loaded elsewhere
\PassOptionsToPackage{unicode}{hyperref}
\PassOptionsToPackage{hyphens}{url}
%
\documentclass[
]{article}
\usepackage{amsmath,amssymb}
\usepackage{iftex}
\ifPDFTeX
  \usepackage[T1]{fontenc}
  \usepackage[utf8]{inputenc}
  \usepackage{textcomp} % provide euro and other symbols
\else % if luatex or xetex
  \usepackage{unicode-math} % this also loads fontspec
  \defaultfontfeatures{Scale=MatchLowercase}
  \defaultfontfeatures[\rmfamily]{Ligatures=TeX,Scale=1}
\fi
\usepackage{lmodern}
\ifPDFTeX\else
  % xetex/luatex font selection
\fi
% Use upquote if available, for straight quotes in verbatim environments
\IfFileExists{upquote.sty}{\usepackage{upquote}}{}
\IfFileExists{microtype.sty}{% use microtype if available
  \usepackage[]{microtype}
  \UseMicrotypeSet[protrusion]{basicmath} % disable protrusion for tt fonts
}{}
\makeatletter
\@ifundefined{KOMAClassName}{% if non-KOMA class
  \IfFileExists{parskip.sty}{%
    \usepackage{parskip}
  }{% else
    \setlength{\parindent}{0pt}
    \setlength{\parskip}{6pt plus 2pt minus 1pt}}
}{% if KOMA class
  \KOMAoptions{parskip=half}}
\makeatother
\usepackage{xcolor}
\setlength{\emergencystretch}{3em} % prevent overfull lines
\providecommand{\tightlist}{%
  \setlength{\itemsep}{0pt}\setlength{\parskip}{0pt}}
\setcounter{secnumdepth}{-\maxdimen} % remove section numbering
\usepackage{bookmark}
\IfFileExists{xurl.sty}{\usepackage{xurl}}{} % add URL line breaks if available
\urlstyle{same}
\hypersetup{
  hidelinks,
  pdfcreator={LaTeX via pandoc}}

\author{}
\date{}

\begin{document}

\subsection{LLM08: Autoridade
Excessiva}\label{llm08-autoridade-excessiva}

\subsubsection{Descrição}\label{descriuxe7uxe3o}

Um sistema baseado em LLM frequentemente é concedido um grau de agência
pelo seu desenvolvedor - a capacidade de interagir com outros sistemas e
realizar ações em resposta a um prompt. A decisão sobre quais funções
invocar também pode ser delegada a um `agente' LLM para determinar
dinamicamente com base no prompt de entrada ou saída do LLM.

A Autoridade Excessiva é a vulnerabilidade que permite a realização de
ações prejudiciais em resposta a saídas inesperadas/ambíguas de um LLM
(independentemente do que está causando o mau funcionamento do LLM; seja
alucinação/confabulação, injeção de prompt direta/indireta, plugin
malicioso, prompts benignos mal projetados ou apenas um modelo com
desempenho ruim). A causa raiz da Autoridade Excessiva geralmente é um
ou mais dos seguintes: funcionalidade excessiva, permissões excessivas
ou autonomia excessiva. Isso difere do tratamento inadequado de saída,
que se preocupa com a falta de escrutínio nas saídas do LLM.

A Autoridade Excessiva pode levar a uma ampla variedade de impactos em
relação à confidencialidade, integridade e disponibilidade, e depende
dos sistemas com os quais um aplicativo baseado em LLM pode interagir.

\subsubsection{Exemplos Comuns desta
Vulnerabilidade}\label{exemplos-comuns-desta-vulnerabilidade}

\begin{enumerate}
\def\labelenumi{\arabic{enumi}.}
\tightlist
\item
  Funcionalidade Excessiva: Um agente LLM tem acesso a plugins que
  incluem funções desnecessárias para a operação pretendida do sistema.
  Por exemplo, um desenvolvedor precisa conceder a um agente LLM a
  capacidade de ler documentos de um repositório, mas o plugin de
  terceiros que eles escolhem também inclui a capacidade de modificar e
  excluir documentos.
\item
  Funcionalidade Excessiva: Um plugin pode ter sido testado durante a
  fase de desenvolvimento e descartado em favor de uma alternativa
  melhor, mas o plugin original permanece disponível para o agente LLM.
\item
  Funcionalidade Excessiva: Um plugin LLM com funcionalidade em aberto
  não filtra corretamente as instruções de entrada para comandos fora do
  necessário para a operação pretendida do aplicativo. Por exemplo, um
  plugin para executar um comando shell específico falha em prevenir
  adequadamente a execução de outros comandos shell.
\item
  Permissões Excessivas: Um plugin LLM tem permissões em outros sistemas
  que não são necessárias para a operação pretendida do aplicativo. Por
  exemplo, um plugin destinado a ler dados se conecta a um servidor de
  banco de dados usando uma identidade que não apenas possui permissões
  SELECT, mas também permissões UPDATE, INSERT e DELETE.
\item
  Permissões Excessivas: Um plugin LLM projetado para realizar operações
  em nome de um usuário acessa sistemas downstream com uma identidade
  genérica de alta privilégio. Por exemplo, um plugin para ler a loja de
  documentos do usuário atual se conecta ao repositório de documentos
  com uma conta privilegiada que tem acesso aos arquivos de todos os
  usuários.
\item
  Autonomia Excessiva: Um aplicativo ou plugin baseado em LLM falha em
  verificar e aprovar independentemente ações de alto impacto. Por
  exemplo, um plugin que permite a exclusão de documentos de um usuário
  realiza exclusões sem qualquer confirmação do usuário.
\end{enumerate}

\subsubsection{Estratégias de Prevenção e
Mitigação}\label{estratuxe9gias-de-prevenuxe7uxe3o-e-mitigauxe7uxe3o}

As seguintes ações podem prevenir a Autoridade Excessiva:

\begin{enumerate}
\def\labelenumi{\arabic{enumi}.}
\tightlist
\item
  Limitar os plugins/ferramentas que os agentes LLM têm permissão para
  chamar apenas às funções mínimas necessárias. Por exemplo, se um
  sistema baseado em LLM não requer a capacidade de buscar o conteúdo de
  uma URL, tal plugin não deve ser oferecido ao agente LLM.
\item
  Limitar as funções implementadas nos plugins/ferramentas LLM para o
  mínimo necessário. Por exemplo, um plugin que acessa a caixa de
  correio de um usuário para resumir e-mails pode precisar apenas da
  capacidade de ler e-mails, então o plugin não deve conter outras
  funcionalidades, como excluir ou enviar mensagens.
\item
  Evitar funções em aberto sempre que possível (por exemplo, executar um
  comando shell, buscar uma URL, etc.) e usar plugins/ferramentas com
  funcionalidades mais granulares. Por exemplo, um aplicativo baseado em
  LLM pode precisar escrever alguma saída em um arquivo. Se isso fosse
  implementado usando um plugin para executar uma função shell, o escopo
  para ações indesejadas seria muito grande (qualquer outro comando
  shell poderia ser executado). Uma alternativa mais segura seria
  construir um plugin de gravação de arquivos que só pudesse suportar
  aquela funcionalidade específica.
\item
  Limitar as permissões concedidas a outros sistemas pelos
  plugins/ferramentas LLM para o mínimo necessário, a fim de limitar o
  escopo de ações indesejadas. Por exemplo, um agente LLM que usa um
  banco de dados de produtos para fazer recomendações de compra a um
  cliente pode precisar apenas de acesso de leitura a uma tabela de
  `produtos'; não deve ter acesso a outras tabelas, nem a capacidade de
  inserir, atualizar ou excluir registros. Isso deve ser aplicado por
  meio da concessão de permissões apropriadas no banco de dados para a
  identidade que o plugin LLM usa para se conectar ao banco de dados.
\item
  Rastrear a autorização do usuário e o escopo de segurança para
  garantir que as ações realizadas em nome de um usuário sejam
  executadas em sistemas downstream no contexto desse usuário específico
  e com as permissões mínimas necessárias. Por exemplo, um plugin LLM
  que lê o repositório de código de um usuário deve exigir que o usuário
  se autentique via OAuth e com o escopo mínimo necessário.
\item
  Utilizar controle humano no loop para exigir que um humano aprove
  todas as ações antes que sejam realizadas. Isso pode ser implementado
  em um sistema downstream (fora do escopo do aplicativo LLM) ou dentro
  do próprio plugin/ferramenta LLM. Por exemplo, um aplicativo baseado
  em LLM que cria e publica conteúdo em mídias sociais em nome de um
  usuário deve incluir um procedimento de aprovação do usuário dentro do
  plugin/ferramenta/API que implementa a operação `publicar'.
\item
  Implementar autorização em sistemas downstream em vez de depender de
  um LLM para decidir se uma ação é permitida ou não. Ao implementar
  ferramentas/plugins, aplicar o princípio de mediação completa para que
  todas as solicitações feitas a sistemas downstream por meio dos
  plugins/ferramentas sejam validadas em relação às políticas de
  segurança.
\end{enumerate}

As seguintes opções não impedirão a Autoridade Excessiva, mas podem
limitar o nível de dano causado:

\begin{enumerate}
\def\labelenumi{\arabic{enumi}.}
\tightlist
\item
  Registrar e monitorar a atividade de plugins/ferramentas LLM e
  sistemas downstream para identificar onde ações indesejadas estão
  ocorrendo e responder adequadamente.
\item
  Implementar limitação de taxa para reduzir o número de ações
  indesejadas que podem ocorrer dentro de um determinado período de
  tempo, aumentando a oportunidade de descobrir ações indesejadas por
  meio de monitoramento antes que ocorra um dano significativo.
\end{enumerate}

\subsubsection{Exemplos de Cenários de
Ataque}\label{exemplos-de-cenuxe1rios-de-ataque}

Um aplicativo de assistente pessoal baseado em LLM recebe acesso à caixa
de correio de um indivíduo por meio de um plugin para resumir o conteúdo
dos e-mails recebidos. Para alcançar essa funcionalidade, o plugin de
e-mail requer a capacidade de ler mensagens, no entanto, o plugin
escolhido pelo desenvolvedor do sistema também inclui funções para
enviar mensagens. O LLM é vulnerável a um ataque de injeção de prompt
indireto, em que um e-mail maliciosamente elaborado engana o LLM para
comandar o plugin de e-mail a chamar a função `enviar mensagens' para
enviar spam da caixa de correio do usuário. Isso poderia ser evitado das
seguintes maneiras: (a) eliminar funcionalidades excessivas, usando um
plugin que oferece apenas capacidades de leitura de e-mails, (b)
eliminar permissões excessivas, autenticando-se no serviço de e-mail do
usuário por meio de uma sessão OAuth com escopo somente leitura, e/ou
(c) eliminar autonomia excessiva, exigindo que o usuário revise
manualmente e clique em `enviar' em cada e-mail redigido pelo plugin
LLM. Alternativamente, o dano causado poderia ser reduzido implementando
limitação de taxa na interface de envio de e-mails.

\subsubsection{Links de Referência}\label{links-de-referuxeancia}

\begin{enumerate}
\def\labelenumi{\arabic{enumi}.}
\tightlist
\item
  \href{https://embracethered.com/blog/posts/2023/chatgpt-cross-plugin-request-forgery-and-prompt-injection./}{Embrace
  the Red: Confused Deputy Problem}: \textbf{Embrace The Red}
\item
  \href{https://github.com/NVIDIA/NeMo-Guardrails/blob/main/docs/security/guidelines.md}{NeMo-Guardrails:
  Interface guidelines}: \textbf{NVIDIA Github}
\item
  \href{https://python.langchain.com/docs/modules/agents/tools/how_to/human_approval}{LangChain:
  Human-approval for tools}: \textbf{Langchain Documentation}
\item
  \href{https://simonwillison.net/2023/Apr/25/dual-llm-pattern/}{Simon
  Willison: Dual LLM Pattern}: \textbf{Simon Willison}
\end{enumerate}

\end{document}
