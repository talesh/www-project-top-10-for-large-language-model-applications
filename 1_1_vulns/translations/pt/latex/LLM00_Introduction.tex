% Options for packages loaded elsewhere
\PassOptionsToPackage{unicode}{hyperref}
\PassOptionsToPackage{hyphens}{url}
%
\documentclass[
]{article}
\usepackage{amsmath,amssymb}
\usepackage{iftex}
\ifPDFTeX
  \usepackage[T1]{fontenc}
  \usepackage[utf8]{inputenc}
  \usepackage{textcomp} % provide euro and other symbols
\else % if luatex or xetex
  \usepackage{unicode-math} % this also loads fontspec
  \defaultfontfeatures{Scale=MatchLowercase}
  \defaultfontfeatures[\rmfamily]{Ligatures=TeX,Scale=1}
\fi
\usepackage{lmodern}
\ifPDFTeX\else
  % xetex/luatex font selection
\fi
% Use upquote if available, for straight quotes in verbatim environments
\IfFileExists{upquote.sty}{\usepackage{upquote}}{}
\IfFileExists{microtype.sty}{% use microtype if available
  \usepackage[]{microtype}
  \UseMicrotypeSet[protrusion]{basicmath} % disable protrusion for tt fonts
}{}
\makeatletter
\@ifundefined{KOMAClassName}{% if non-KOMA class
  \IfFileExists{parskip.sty}{%
    \usepackage{parskip}
  }{% else
    \setlength{\parindent}{0pt}
    \setlength{\parskip}{6pt plus 2pt minus 1pt}}
}{% if KOMA class
  \KOMAoptions{parskip=half}}
\makeatother
\usepackage{xcolor}
\setlength{\emergencystretch}{3em} % prevent overfull lines
\providecommand{\tightlist}{%
  \setlength{\itemsep}{0pt}\setlength{\parskip}{0pt}}
\setcounter{secnumdepth}{-\maxdimen} % remove section numbering
\usepackage{bookmark}
\IfFileExists{xurl.sty}{\usepackage{xurl}}{} % add URL line breaks if available
\urlstyle{same}
\hypersetup{
  hidelinks,
  pdfcreator={LaTeX via pandoc}}

\author{}
\date{}

\begin{document}

OWASP Top 10 para aplicações LLM

Versão 1.0

Publicado: em 31 de dezembro de 2023

HTTPS://LLMTOP10.COM

\subsection{Introdução}\label{introduuxe7uxe3o}

\subsubsection{A Origem da Lista}\label{a-origem-da-lista}

O frenesi e interesse nos Modelos de Linguagem de Grande Porte (LLMs)
após os chatbots pré-treinados do mercado em massa no final de 2022 tem
sido notável. Empresas, ansiosas para aproveitar o potencial dos LLMs,
estão rapidamente integrando-os em suas operações e ofertas de solução
voltadas para os clientes. No entanto, a velocidade vertiginosa com que
os LLMs estão sendo adotados superou o estabelecimento de protocolos de
segurança abrangentes, deixando muitas aplicações vulneráveis a
problemas de alto risco.

A ausência de um recurso unificado que aborde essas preocupações de
segurança nos LLMs era evidente. Desenvolvedores, não familiarizados com
os riscos específicos associados aos LLMs, ficaram com recursos
dispersos e a missão da OWASP parece se encaixar perfeitamente para
ajudar a promover uma adoção mais segura dessa tecnologia.

\subsubsection{Publico-Alvo?}\label{publico-alvo}

Nosso público-alvo é composto por desenvolvedores, cientistas de dados e
especialistas em segurança encarregados de projetar e construir
aplicativos e plug-ins utilizando as tecnologias LLM. Nosso objetivo é
fornecer orientações de segurança práticas, aplicáveis e concisas que
ajudem esses profissionais a navegar no complexo terreno da segurança em
LLM que está em constante evolução

\subsubsection{A Criação da Lista}\label{a-criauxe7uxe3o-da-lista}

A criação da lista OWASP Top 10 para LLMs foi um projeto muito
importante, construído com base na expertise coletiva de uma equipe
internacional de quase 500 especialistas, com mais de 125 colaboradores
ativos. Nossos colaboradores vêm de diferentes origens, incluindo
empresas de IA (Inteligência Artificial), empresas de segurança, ISVs
(Fornecedor de Software Independente), provedores de serviço na nuvem,
fornecedores de hardware e setor acadêmico.

Ao longo de um mês, discutimos e propusemos vulnerabilidades potenciais,
com membros da equipe escrevendo 43 ameaças distintas. Por meio de
várias rodadas de votação, refinamos essas propostas para uma lista
concisa das dez vulnerabilidades mais críticas. Cada vulnerabilidade foi
então examinada e refinada por subequipes dedicadas e submetida a uma
revisão pública, garantindo a lista final que é mais abrangente e
aplicável.

Cada uma dessas vulnerabilidades, juntamente com exemplos comuns, dicas
de prevenção, cenários de ataque e suas referências, foram examinadas e
refinadas ainda mais por subequipes dedicadas e submetidas a revisão
pública, garantindo a lista final que é mais abrangente e aplicável.

\subsubsection{Relação com outras Listas OWASP Top
10}\label{relauxe7uxe3o-com-outras-listas-owasp-top-10}

Embora esta lista compartilhe características com tipos de
vulnerabilidades encontradas em outras listas OWASP Top 10, não
reiteramos simplesmente essas vulnerabilidades. Em vez disso,
aprofundamos nas implicações que são únicas à essas vulnerabilidades e a
relação que elas têm ao serem encontradas em aplicações que utilizam os
LLMs.

Nosso objetivo é unir os princípios gerais de segurança de aplicações
com os desafios específicos apresentados pelos LLMs. Isso inclui
explorar como vulnerabilidades convencionais podem representar riscos
diferentes ou serem exploradas de maneiras novas nos LLMs, além de como
as estratégias tradicionais de remediação de problemas precisam ser
adaptadas para aplicações que utilizam os LLMs.

\subsubsection{Sobre a Versão 1.1}\label{sobre-a-versuxe3o-1.1}

\subsubsection{Steve Wilson}\label{steve-wilson}

Líder do Projeto, OWASP Top 10 para Aplicações de IA LLM\\
\href{https://www.linkedin.com/in/wilsonsd/}{https://www.linkedin.com/in/wilsonsd}\\
Twitter/X: @virtualsteve

\subsubsection{Ads Dawson}\label{ads-dawson}

v1.1 release Lead \& Vulnerability Entries Lead, OWASP Top 10 para
Aplicações de IA LLM\\
\href{https://www.linkedin.com/in/adamdawson0/}{https://www.linkedin.com/in/adamdawson0}\\
GitHub:@GangGreenTemperTatum

\subsection{Sobre esta tradução}\label{sobre-esta-traduuxe7uxe3o}

\subsubsection{Versão 1.1 Colaboradores da Tradução para o Português
Brasileiro}\label{versuxe3o-1.1-colaboradores-da-traduuxe7uxe3o-para-o-portuguuxeas-brasileiro}

\begin{itemize}
\tightlist
\item
  \textbf{Emmanuel Guilherme Junior}\\
  \url{https://www.linkedin.com/in/emmanuelgjr/}\strut \\
\item
  \textbf{Rubens Zimbres}\\
  \url{https://www.linkedin.com/in/rubens-zimbres/}
\end{itemize}

Reconhecendo a natureza excepcionalmente técnica e crítica do OWASP Top
10 para Aplicações de IA LLM, optamos conscientemente por empregar
apenas tradutores humanos na criação desta tradução. Os tradutores
listados acima não só possuem um profundo conhecimento do conteúdo
original, mas também a fluência necessária para tornar esta tradução um
sucesso.

Talesh Seeparsan\\
Líder de Tradução, OWASP Top 10 para Aplicações de IA LLM\\
\url{https://www.linkedin.com/in/talesh/}

\subsection{OWASP Top 10 para Aplicações de IA
LLM}\label{owasp-top-10-para-aplicauxe7uxf5es-de-ia-llm}

\subsubsection{LLM01: Injeção de
Prompt}\label{llm01-injeuxe7uxe3o-de-prompt}

Isso manipula o modelo de linguagem de grande porte (LLM) por meio da
manipulação de prompt, gerando ações não intencionais pelo LLM. Injeções
diretas sobrescrevem prompts do sistema, enquanto as indiretas manipulam
entradas de fontes externas para um resultado determinado. \#\#\# LLM02:
Manipulação Insegura de Output Essa vulnerabilidade ocorre quando o
output de um LLM é aceito sem ser escrutinado, expondo os sistemas de
backend. O uso indevido pode levar a consequências graves, como XSS,
CSRF, SSRF, escalonamento de privilégios ou execução remota de código.
\#\#\# LLM03: Envenenamento dos Dados de Treinamento Essa
vulnerabilidade ocorre quando os dados de treinamento do LLM são
adulterados, introduzindo vulnerabilidades ou vieses que comprometem a
segurança, eficácia ou comportamento ético. As fontes incluem Common
Crawl, WebText, OpenWebText e livros. \#\#\# LLM04: Negação de Serviço
ao Modelo Adversários geram operações intensivas nos recursos dos LLMs,
levando à degradação do serviço ou a custos muito elevados. A
vulnerabilidade é ampliada devido à natureza intensiva por parte de
recursos dos LLMs e a imprevisibilidade de entradas pelo usuário. \#\#\#
LLM05: Vulnerabilidades na Cadeia de Suprimentos O ciclo de vida de
aplicativos LLM pode ser comprometido por componentes ou serviços
vulneráveis, levando a ataques na segurança. O uso de conjuntos de dados
advindo de terceiros, modelos pré-treinados e plug-ins podem adicionar
vulnerabilidades. \#\#\# LLM06: Divulgação de Informações Sensíveis Os
LLMs podem inadvertidamente revelar dados confidenciais em suas
respostas, levando ao acesso não autorizado dos dados, violações de
privacidade e violações de segurança. É crucial implementar a
higienização dos dados e implementar políticas de usuários rigorosas
para mitigar isso. \#\#\# LLM07: Design Inseguro de Plug-ins Os plug-ins
dos LLMs podem ter entradas inseguras e controle de acesso insuficiente.
Essa falta de controle do aplicativo torna-os mais fáceis de explorar e
pode resultar em consequências como execução remota de código. \#\#\#
LLM08: Autoridade Excessiva Sistemas baseados nas LLMs podem realizar
ações que levam a consequências não intencionais. O problema surge da
funcionalidade excessiva, permissões ou autonomia concedidas aos
sistemas baseados nas LLMs. \#\#\# LLM09: Dependência Excessiva Sistemas
ou pessoas excessivamente dependentes nas LLMs sem supervisão podem
enfrentar desinformação, má comunicação, problemas legais e
vulnerabilidades de segurança devido a conteúdo incorreto ou inadequado
gerado pela LLMs. \#\#\# LLM10: Roubo do Modelo Isso envolve acesso não
autorizado, cópia ou vazamento de modelos de LLM proprietários. O
impacto inclui perdas econômicas, comprometimento da vantagem
competitiva e potencial acesso a informações sensíveis.

\end{document}
